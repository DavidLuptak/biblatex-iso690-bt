\documentclass{csbulletin}
\usepackage[utf8]{inputenc}
\usepackage{minted}  %% Highlighted source code for LaTeX

\newcommand{\cmdmint}[1]{\mintinline{latex}{#1}}

\usepackage[style=iso-numeric]{biblatex}
\addbibresource{../resource.bib}

% Hack for hyperref -- from csbulletin template
% \mubytein 0
\usepackage{hyperref}
\hypersetup{
    pdftitle = {Sadzba bibliografie podľa normy ISO 690 v~systéme
      \LaTeX{}},
    pdfauthor = {Dávid Lupták},
    pdfkeywords = {ISO 690, bibliografia, citácie, biblatex}
}

\setcounter{secnumdepth}{2}

\begin{document}
\selectlanguage{slovak}

\title{Sadzba bibliografie podľa normy ISO 690 v~systéme \LaTeX}
\EnglishTitle{Typesetting of bibliography in \LaTeX{} complying
  with the international standard ISO 690}
\author{Dávid Lupták}
\podpis{Dávid Lupták\\
  \href{mailto:422640@mail.muni.cz}{422640@mail.muni.cz}\\
  Fakulta informatiky, Masarykova univerzita, Brno}
\maketitle

\begin{abstract}
Dodržiavanie normy ISO 690 pri tvorbe bibliografických odkazov
a citácií býva vyžadované mnohými inštitúciami nielen v~českom
akademickom prostredí. V~systéme \LaTeX{} však doteraz neexistovala
žiadna podpora, ktorá by plnohodnotne riešila túto problematiku. Až
na základe referenčnej implementácie balíka \textsf{biblatex-iso690}
vznikol balíček, ktorý splňuje požiadavky normy v~plnom rozsahu
a výrazne tak zjednodušuje citovanie informačných zdrojov.
\end{abstract}

\medskip\par\textbf{Kľúčové slová}\smallskip\par
ISO 690, bibliografia, citácie, biblatex

\begin{refsection}

\section{Úvod}

Odborné články vyžadujú vysoký stupeň práce s~inými odbornými textami
a informačnými zdrojmi, na ktoré je potrebné korektne sa odkazovať a
tieto zdroje správne citovať. V~českom akademickom prostredí prevláda
tvorba bibliografických odkazov a citácií podľa normy
ISO 690~\cite{Kratochvil2011}. Úvodom tohoto článku je v~stručnosti
predstavená norma ISO 690 a takisto aj pre ňu už existujúce
implementácie. Osobitne sú uvedené možnosti sadzby bibliografie
v~systéme \LaTeX{} a v~záverečnej časti nasleduje oboznámenie sa
s~balíkom \textsf{biblatex-iso690}, prvou úplnou \LaTeX{}ovou
podporou aktuálne platnej normy ISO 690.


\section{Norma ISO 690}

Tvorba bibliografických odkazov a citácií sa v~minulosti riadila
pravidlami noriem ISO 690:1987~\cite{iso690:1987} pre tlačené
informačné zdroje a ISO 690-2:1997~\cite{iso690:2:1997} pre zdroje
elektronické. V~roku 2010 boli tieto dve normy zjednotené a nahradené
novou verziou normy ISO 690:2010~\cite{iso690:2010}. Normalizačné
organizácie (členovia ISO) zabezpečujú preklady noriem na národnej
úrovni~\cite{isoMembershipManual}. Takýmto prekladom bola v~roku 2011
prijatá aj česká verzia normy ČSN ISO 690:2011~\cite{csn:iso690:2011}
alebo v~roku 2012 slovenská technická norma
STN ISO 690:2012~\cite{stn:iso690:2012}. Tieto preklady nadobúdajú
rovnaký status ako oficiálna verzia normy ISO 690:2010. Aj napriek
tomu obsahuje napríklad česká norma ČSN ISO 690:2011 množstvo
nejasných miest a nepresností. Táto problematika je však prenechaná
na iné články Zpravodaja~\cite{Hala2013}.

\subsection{Terminológia}

K~úspešnému porozumeniu nasledovného textu je potrebné uviesť dva
základné termíny, ktoré častokrát bývajú medzi sebou
zamieňané\footnote{potvrdzuje to aj článok o~predstavení originálu
normy, ktorý bol napísaný ešte pred vznikom českej verzie
normy~\cite{Bratkova2010}; takisto interpretácia normy od tej istej
autorky~\cite{Bratkova2011}}~\cite{csn:iso690:2011}:

\begin{description}
\item[odkaz] údaj v~texte alebo iný druh obsahu dokumentu na
  príslušnú bibliografickú citáciu
\item[citácia] dáta popisujúce informačný zdroj alebo jeho časť,
  dostatočne presne a podrobne na to, aby mohol byť tento zdroj
  identifikovaný a bolo možné ho vyhľadať
\end{description}

\subsection{Zásada konzistencie}

Norma ISO 690 hneď v~úvode svojho výkladu priamo uvádza, že nemá za
cieľ definovať konkrétny štýl bibliografického odkazu alebo citácie.
Použitý štýl a interpunkcia v~ilustračných príkladoch nie sú súčasťou
doporučenia. Tento fakt prináša dve zásadné poznatky:

\begin{enumerate}
\item norma ctí princíp oddelenia formy od obsahu
\item normu nemožno považovať za citačný štýl~\cite{Hala2013}
\end{enumerate}

\noindent Zároveň však norma doporučuje, aby bol pre všetky citácie
v~dokumente použitý jednotný štýl, formát a interpunkcia. Táto
požiadavka však už zostáva na samotnom upravovateľovi citácie, ktorý
môže čerpať z~ilustračných príkladov v~norme samotnej, interpretácií
alebo iných zaužívaných zvyklostí sadzby bibliografie.


\section{Sadzba bibliografie v~systéme \LaTeX{}}

V~systéme \LaTeX{} sú k~dispozícii prakticky tri základné prístupy na
sadzbu bibliografie~\cite{talbot2013}. Prvým z~nich je použitie
čistého \LaTeX{}u, zvyšné dva uznávajú princíp oddelenia formy od
obsahu a na sadzbu bibliografie používajú externú bibliografickú
databázu a takisto aj externý program na preklad.

\subsection{Čistý \LaTeX}\label{bib:pure:latex}

% \subsubsection{Použitie}

\LaTeX{} poskytuje na sadzbu bibliografie vstavané prostredie
\texttt{thebibliography} a rodinu makier \cmdmint{\cite} umožňujúcich
sadzbu bibliografických odkazov. Tie následne odkazujú na samotné
bibliografické citácie. Zoznam bibliografických citácií je uvedený
v~prostredí \texttt{thebibliography}, jednotlivé položky sú uvedené
príkazom \cmdmint{\bibitem}.

\begin{minted}[frame=lines]{latex}
\documentclass{...}
\begin{document}
\cite{<identifikátor01>}
...
\begin{thebibliography}{<maximálna šírka identifikátora>}
\bibitem{<identifikátor01>}
  <Autor>. \emph{<Názov>: <podnázov>}. ...
...
\end{thebibliography}
\end{document}
\end{minted}

% \subsubsection{Nevýhody}

Predchádzajúca ukážka kódu okrem základnej syntaxe poukazuje aj na
fakt, že takýto prístup nie je vhodný na sadzbu bibliografie
rozsiahlejších diel~\cite{talbot2012}. Prináša tieto hlavné nevýhody:

\begin{enumerate}
\item vysádzané sú všetky citácie uvedené v~zozname prostredia
  \texttt{thebibliography} (bez ohľadu na to, či boli vôbec
  citované)\label{item:1}
\item každý záznam je nutné naformátovať jednotlivo (v~závislosti
  od požadovaného bibliografického štýlu)\label{item:2}
\item bibliografické citácie sú radené presne v~takom poradí ako sú
  uvedené v~zozname prostredia \texttt{thebibliography}\label{item:3}
\end{enumerate}

\noindent Okolnosti ohľadom nevýhody číslo~\ref{item:1} norma ISO 690
priamo nešpecifikuje, takýto prístup však nenasleduje všeobecné
odporúčania tvorby bibliografie~\cite{talbot2013}. Ďalej je v~prípade
prostredia \texttt{thebibliography} ťažké zaručiť zásadu jednotnosti
citácií (podľa~\ref{item:2}) a vôbec nie je možné zaistiť správne
poradie citácií (podľa~\ref{item:3}) pre ktorúkoľvek prípustnú metódu
citovania špecifikovanú v~norme ISO 690.

% \subsubsection{Kompilácia}

Znovupoužiteľnosť bibliografických záznamov a škálovateľnosť zoznamu
citácií nie je silnou stránkou tohoto riešenia. Výhodou je však
rýchlosť a nízky počet prekladov dokumentu (postačí dvojnásobný
preklad \TeX ovým kompilátorom).


%%%%%%%%%%%%% BIBTEX %%%%%%%%%%%%%%%%%%%%%%%

\subsection{B\textsc{ib}\TeX{}}

Preferovaným spôsobom práce s~bibliografiou pri sadzbe rozsiahlejších
typov dokumentov je vytvorenie externej databázy bibliografických
záznamov (\ref{bib:file}) a použitie externého programu na jej
preklad~\cite{talbot2013}. Tento externý program zaistí správne
zoradenie citácií (rieši problém~\ref{item:3}) a podľa použitého
bibliografického štýlu\footnote{\label{footnote:ambiguous:style:names}
často nazývaný aj citačný štýl, v~tomto kontexte by bolo možné
hovoriť všeobecne o~formátovacom štýle} (rieši problém~\ref{item:2})
vytvorí formátované \LaTeX{}ové prostredie \texttt{thebibliography}
s~bibliografiou na vysadenie. Typickým zástupcom tohoto prístupu je
\BibTeX{}, ktorého nespornou výhodou je práve spomínané oddelenie
formy od obsahu.

% \subsubsection{Použitie}

Príkaz \cmdmint{\bibliographystyle} slúži na definovanie
formátovacieho štýlu, príkaz \cmdmint{\bibliography} určuje, ktoré
bibliografické databázy sa majú použiť a takisto miesto vysadenia
v~zdrojovom dokumente. Príkazom \cmdmint{\cite} sa vytvorí odkaz
v~texte dokumentu na danú citáciu. Je možné použiť aj príkaz
\cmdmint{\nocite}, ktorý nevytvorí odkaz v~texte samotného dokumentu,
zaručí však výskyt citácie v~zozname bibliografických citácií (rieši
problém~\ref{item:1}).

\begin{minted}[frame=lines]{latex}
\documentclass{...}
\bibliographystyle{<formátovací štýl>}
\begin{document}
\cite[<text>]{<zoznam identifikátorov>}
...
\bibliography{<databáza01>,<databáza02>,...}
\end{document}
\end{minted}

% \subsubsection{Nevýhody}

Spomenuté výhody však vyvažuje, resp. prevyšuje, množstvo nevýhod
spojených so skutočnosťou, že vývoj \BibTeX u~má relatívne stagnujúcu
tendenciu~\cite{patashnik1994,patashnik2003}. Medzi hlavné nevýhody
patrí:

\begin{enumerate}
\item problémy so vstupným kódovaním~\cite{ctan:bibtex} (hoci je
  dostupné alternatívne
  riešenie)\footnote{\url{https://www.ctan.org/pkg/bibtex8bit}}
\item zložitosť tvorby vlastných formátovacích
  štýlov~\cite{patashnik1988:styles} (hoci je dostupné riešenie na
  automatizované generovanie
  štýlov)\footnote{\url{https://www.ctan.org/pkg/custom-bib}}
\item výkonnostné problémy (pretečenie pamäti pri práci s~veľkými
  bibliografickými databázami)~\cite{biblatex:manual2016}
\item slabá podpora tvorby odkazov v~texte dokumentu~\cite{bibtex:faq}
  (hoci sú dostupné flexibilnejšie
  možnosti)\footnote{\url{https://www.ctan.org/pkg/natbib},
  \url{https://www.ctan.org/pkg/cite}}
\item absencia dnes už štandardných polí, napr. poľa \texttt{url}
  (hoci sú dostupné alternatívne
  riešenia)\footnote{\url{https://www.ctan.org/pkg/natbib},
  \url{https://www.ctan.org/pkg/babelbib}}
\item chýbajúca podpora lokalizácie a viacjazyčnosti
  citácií~\cite{Harders2002} (hoci je riešenie
  dostupné)\footnote{\url{https://www.ctan.org/pkg/babelbib}}
\end{enumerate}

% \subsubsection{Kompilácia}

Na korektné vysádzanie zdrojového dokumentu sú potrebné minimálne
tri preklady \TeX ovým kompilátorom a jeden programom \BibTeX{}.
Globálne aplikovateľná schéma na sadzbu bibliografie pomocou
\BibTeX{}u je nasledovná~\cite{Markey2009}:

\begin{center}
\LaTeX\ (\BibTeX\ \LaTeX)$^+$ \LaTeX\
\end{center}

\noindent Oproti sadzbe bibliografie bez použitia externého programu
je síce počet potrebných prekladov vyšší, na druhej strane však táto
\uv{zložitosť} prináša riešenia na takmer všetky vyššie spomenuté
problémy spojené s~použitím iba čistého \LaTeX{}u.


%%%%%%%%%%%%% BIBLATEX %%%%%%%%%%%%%%%

\subsection{BibLaTeX}\label{biblatex:typesetting}

Inou možnosťou externého programu na sadzbu bibliografie je
modernejší a flexibilnejší \LaTeX ový balík BibLaTeX. Tento balíček
je kompletnou reimplementáciou prostriedkov na prácu s~bibliografiou
v~systéme \LaTeX{}, často označovaný aj ako nástupca
\BibTeX{}u~\cite{ctan:bibtex,hufflen2011}. Na formátovanie záznamov
používa výhradne \LaTeX ové makrá a na spracovanie bibliografickej
databázy (\ref{bib:file}) a jednotlivých položiek nástroj
Biber~\cite{biblatex:manual2016}.

% \subsubsection{Použitie}

Použitie balíka BibLaTeX je mierne odlišné od tradičného \BibTeX{}u.
Príkazy majú nielen odlišnú syntax, ale poskytujú aj rozsiahlejšie
možnosti práce s~bibliografickými záznamami. Formátovací štýl (je
možné špecifikovať bibliografický a citačný štýl zvlášť) sa definuje
priamo ako voľba balíka pri jeho načítaní, t.\,j. ako voliteľný
argument príkazu \cmdmint{\usepackage}. Na špecifikovanie
bibliografickej databázy slúži príkaz \cmdmint{\addbibresource}, kde
je potrebné uviesť celý názov súboru aj s~príponou \cmdmint{.bib}.
Na vysádzanie samotnej bibliografie v~texte slúži príkaz
\cmdmint{\printbibliography}, ktorý sa v~dokumente môže vyskytovať
\emph{aj} viackrát. Samozrejme na tvorbu odkazu v~texte dokumentu
slúži príkaz \cmdmint{\cite} a jeho varianty. Nasledujúca ukážka kódu
pokrýva len základnú štruktúru dokumentu, použitie príkazov je omnoho
komplexnejšie:

\begin{minted}[frame=lines]{latex}
\documentclass{...}
\usepackage[...]{biblatex}
\addbibresource{<databáza01.bib>}
\addbibresource{<databáza02.bib>}
\begin{document}
\cite{...}
...
\printbibliography
\end{document}
\end{minted}

% \subsubsection{Výhody a nevýhody}

Balík BibLaTeX rieši množstvo problémov uvedených v~prípade
\BibTeX{}u. Medzi tie najvýznamnejšie patrí~\cite{ctan:biblatex}:

\begin{enumerate}
\item plná podpora kódovania Unicode
\item pokročilé možnosti radenia (použité technológie
  \emph{Unicode Collation Algorithm} a
  \emph{Unicode Common Locale Data Repository} (CLDR))
\item viacjazyčnosť záznamov (balíky \textsf{babel} a
  \textsf{polyglossia})
\item rozšírený formát bibliografických záznamov
\item veľké množstvo dostupných štýlov
\item flexibilné vytváranie nových štýlov
\end{enumerate}

\noindent Samozrejme, tento zoznam je len výňatok z~bohatej
funkcionality BibLaTeXu~\cite{biblatex:manual2016}. Dôkazom toho je
aj minimálny počet nevýhod tohoto balíka. Za všetky je možné spomenúť
nekompatibilitu medziformátu sadzby bibliografie s~pôvodným
\BibTeX{}ovým riešením~\cite{stack:ex:biblatex:drawback}, ktorý býva
vyžadovaný pri predkladaní vydavateľovi.

% \subsubsection{Kompilácia}

Preklad prebieha analogicky ako u~\BibTeX{}u. Najskôr sa prekladá
\TeX{}ovým kompilátorom, následne sa použije spomínaný nástroj Biber
nad vygenerovaným súborom \cmdmint{.bcf} a nakoniec je potrebný ešte
jeden preklad kompilátorom \TeX u. Schéma sadzby dokumentu za
použitia BibLaTeXu vyzerá nasledovne\footnote{koncovky súborov nie je
potrebné uvádzať}:

\begin{minted}[frame=lines]{bash}
latex <dokument>[.tex]
biber <dokument>[.bcf]
latex <dokument>[.tex]
\end{minted}


%%%%%%%%%%%%% BIB SUBOR %%%%%%%%%%%%%%%

\subsection{Databáza bibliografických záznamov (\texttt{.bib} súbor)}\label{bib:file}

Pre kompletnosť tejto kapitoly je žiadúce uviesť aj samotný formát
bibliografickej databázy. Tá obsahuje jednotlivé bibliografické
záznamy, každý takýto záznam má svoj typ, jedinečný identifikátor a
(typicky) niekoľko dvojíc \emph{pole--hodnota} definujúce práve
samotné bibliografické údaje. Všeobecný vzor takéhoto záznamu vyzerá
nasledovne:

\begin{minted}[frame=lines]{latex}
@<typ záznamu>{identifikátor,
  <názov poľa> = {hodnota},
  ...
  <názov poľa> = {hodnota}
}
\end{minted}

\noindent 

\noindent Všetky podporované typy záznamov \BibTeX{}u sú
aplikovateľné aj pre balík BibLaTeX, a to buď priamo alebo
definovaním aliasu. Rozdielom je akurát to, že BibLaTeX oproti
\BibTeX{}u definuje naviac ešte niekoľko ďalších typov.

Obdobná situácia je aj v~prípade polí záznamov. Balík BibLaTeX
poskytuje spätnú kompatibilitu pre všetky polia dostupné v~\BibTeX{}u
a k~tomu ponúka ďalšie. Okrem \emph{bežných} polí sú to aj polia
\emph{špeciálne}, ktoré slúžia napríklad na nastavenie jazyka daného
bibliografického záznamu pre podporu viacjazyčnosti.

\subsection{Zhrnutie}

Na jednoduchú sadzbu bibliografie jednorázového dokumentu s~malým
počtom citácií je najvýhodnejšie použiť vstavanú funkcionalitu
\LaTeX{}u. Pri rozsiahlejších dokumentoch s~väčším počtom citácií je
však už výhodné použiť služby externých programov na sadzbu
bibliografie. Takýmto prístupom oddelenia formy od obsahu
(bibliografická databáza sa nachádza v~samostatnom súbore,
formátovacie štýly taktiež osobitne) je možné dosiahnuť vysokú
škálovateľnosť, znovupoužiteľnosť bibliografických záznamov a
flexibilnú a efektívnu manipuláciu s~citáciami.

Okrem samotného \BibTeX u~existuje množstvo programov na ňom
založených, ich problémom však je práve skutočnosť, že vychádzajú
z~\BibTeX{}u. Týka sa to najmä použitia formátovacích štýlov, hoci
niektoré sa pokúšajú o~nahradenie jazyka BST\footnote{\BibTeX{}
bibliografický štýl} iným, modernejším, programovacím jazykom (zväčša
XML)~\cite{hufflen2011,hufflen2008}.

Spomedzi dostupných možností sadzby bibliografie v~systéme
\LaTeX{}~\cite{talbot2013,Mittelbach2004} vychádza ako najlepšia
voľba balík BibLaTeX s~nástrojom Biber na
preklad~\cite{hufflen2011,biber:manual2016}.


\section{Existujúce riešenia pre ISO 690}

Táto sekcia uvádza niekoľko existujúcich implementácií, ktoré
zahŕňajú podporu pre dodržiavanie normy ISO 690. Prvé dve z~nich sú
priamo použiteľné v~sádzacom systéme \LaTeX{}, jazyk CSL je
predstavený z~dôvodu jeho súčasnej popularity a balík OPmac-bib ako
jeden z~mála dostupných balíkov poskytujúcich plnohodnotnú podporu
tvorby bibliografických odkazov a citácií.

\subsection{czechiso}

Pre české normy ČSN ISO 690:1996~\cite{csn:iso690:1996} a
ČSN ISO 690-2:2000~\cite{csn:iso690:2:2000} existuje pre \BibTeX{}
neoficiálny formátovací štýl \textsf{czechiso} z~roku 2006. Autorom
je David Martinek a tento štýl je dostupný na adrese
\url{http://www.fit.vutbr.cz/~martinek/latex/czechiso.html}. Táto
implementácia nezodpovedá normám presne, niektoré vyžadované položky
bibliografických záznamov absentujú, použité funkcie by bolo vhodné
prepísať aby splňovali požiadavky normy.

\subsection{biblatex-iso690}

V~roku 2011 vznikla prvá referenčná implementácia bibliografického a
citačného štýlu pre balík BibLaTeX podľa normy ISO 690. Tá však
vychádzala z~predošlých verzií
noriem~\cite{csn:iso690:1996,csn:iso690:2:2000} a prevažne sa
pridržiavala českej interpretácie~\cite{Bratkova2008}. Autorom je
Michal Hoftich a štýl bol sprístupnený ako neoficiálna verzia na
adrese \url{https://github.com/michal-h21/biblatex-iso690}. Podobne
ako v~predchádzajúcom prípade, ani táto implementácia nezodpovedá
normám presne, na domovskej stránke projektu je zaznamenaných
niekoľko nahlásených problémov ohľadom funkcionality a použitia
tohoto štýlu.

\subsection{Jazyk CSL}

Jazyk CSL je programovací jazyk založený na jazyku XML. Populárnym
sa stal s~vydaním Zotera v~roku 2006~\cite{Fenner2010}.

Medzi jednoznačné výhody patrí syntax jazyka XML. Na to nadväzuje
obľúbenosť tohoto formátu a otvorenosť a univerzálnosť jazyka
CSL~\cite{Ansorge2013}. Nespornou výhodou je aj jeho využiteľnosť
naprieč viacerými aplikáciami, čo dokazuje aj rozsiahly zoznam
produktov na oficiálnych stránkach projektu CSL, ktoré tento jazyk
využívajú~\cite{csl:home}. Patrí medzi ne napríklad Zotero, Papers
či Mendeley.

V~repozitári projektu \emph{Citation Style Language}~\cite{csl:home}
sú medzi obrovským množstvom citačných štýlov dostupné aj tie podľa
normy ISO 690, v~celkovom počte 15 štýlov. Z~tohoto počtu je každý
pre inú jazykovú lokalizáciu alebo metódu citovania. V~každom z~nich
sa nájdu aj drobné odchýlky alebo nepresnosti voči výkladu normy
ISO 690. Vo všeobecnosti existuje pre jazyk CSL tento rad
nevýhod~\cite{csl:styles}:

\begin{itemize}
\item nie je možné nastaviť formát hodnoty identifikátora
\item obmedzená podpora pre legislatívne štýly (Multilingual Zotero
  môže byť riešením)
\item obmedzená podpora pre viacjazyčnosť citácií (Multilingual
  Zotero môže byť riešením)
\item nie je možné zadať rozsah dátumu do poľa pre dátum (údaj sa
  nevygeneruje)
\end{itemize}

Na záver je potrebné dodať, že napríklad balík BibLaTeX všetky tieto
problémy pokrýva vo svojej základnej
funkcionalite~\cite{biblatex:manual2016}.

\subsection{OPmac-bib}

OPmac je sada makier umožňujúca pohodlnejšiu prácu s~plain\TeX{}om,
ktoré poskytujú základnú \LaTeX{}ovú funkcionalitu. Balíček OPmac-bib
je nadstavbou týchto makier, zaoberajúci sa práve sadzbou
bibliografie. Oproti ostatným nástrojom však nepoužíva žiaden externý
program, ale celú funkcionalitu rieši na úrovni \TeX{}ových makier.
Autorom balíka je Petr Olšák a k~dispozícii je od roku 2015 v~rámci
balíka \textit{csplain}.

% \subsubsection{Nové polia v~\texttt{.bib} súboroch}

OPmac-bib pracuje priamo s~\texttt{.bib} súbormi a je akousi
nadstavbou a rozšírením nad starodávnym \BibTeX om. Využíva teda
všetky typy a polia, ktoré sú dostupné v~\BibTeX{}u~a naviac prináša
nové polia, ktoré sú v~dnešnej dobe nutnosťou (vo všeobecnosti ako
aj pre štýl ISO 690). Ide napríklad o~polia

\begin{center}
\texttt{url},\enspace \texttt{doi}\enspace alebo\enspace \texttt{lang},
\end{center}

\noindent vďaka ktorým už nemusí autor \texttt{.bib} súboru vkladať
tieto informácie do poľa \texttt{note}, ale do príslušných, na tento
účel vytvorených, polí. Tým sa zvyšuje flexibilita a v~konečnom
dôsledku aj možnosť korektného výpisu poradia údajov podľa aktuálnej
verzie normy.

Tým, že balíček OPmac-bib úplne obchádza použitie externého programu
a číta \texttt{.bib} databázu priamo pomocou makier \TeX{}u, zvyšuje
čitateľnosť kódu a zároveň umožňuje jednoduchšie predefinovanie
makier v~prípade nutnosti ich prispôsobenia špecifickým potrebám.
Norma ISO 690 totiž vynucuje uvádzanie niektorých údajov, pre ktoré
v~pôvodnom \BibTeX{}u~príslušné polia neexistujú. Vytváranie nových
polí pre každý takýto špecifický údaj však nie je riešenie. OPmac to
z~tohoto hľadiska rieši vskutku elegantne, obdobne ako balík
BibLaTeX. Poskytuje univerzálne polia, do ktorých možno uvádzať
nielen bibliografické údaje, ale zároveň aj makrá používané na výpis
prvkov a tým docieliť požadovanú podobu výstupu. Ide napríklad
o~polia

\begin{center}
\texttt{option}\enspace a\enspace \texttt{ednote}.
\end{center}

Pole \texttt{option} je možné použiť napríklad v~prípade potreby
uvedenia dodatočného názvu diela, jeho prekladu apod. Vďaka takýmto
voľbám je možné dosiahnuť adekvátny výpis údajov splňujúci aktuálnu
verziu normy ISO 690.

Pole \texttt{ednote} slúži na uvedenie vedľajších tvorcov alebo na
rôzne doplňujúce informácie. Bežne sa tu môžu vyskytnúť údaje, ktoré
nemožno jednoducho algoritmizovať, preto je potrebné do tohoto poľa
zadať údaje v~takej podobe, v~akej sa majú objaviť na výstupe.
Typickým príkladom môžu byť informácie o~prekladateľovi alebo
pôvodcoch ďalších vydaní.

Vďaka flexibilným poliam záznamov poskytuje balík OPmac-bib plnú
podporu normy ISO 690 a preto je jasnou voľbou pre použitie pri práci
s~plain\TeX{}om.
% Drobným nedostatkom tohoto balíka je interoperabilita \texttt{.bib}
% súboru, ktorý pre použitie v OPmac-bib môže obsahovať značne
% špecifické polia záznamov, ktoré nebudú použiteľné pre žiadne iné
% implementácie. Momentálne však OPmac-bib obsahuje aj také polia,
% ktoré by sa dali jednoducho zmeniť, aby sa miera interoperability
% medzi rôznymi implemntáciami aspoň mierne zvýšila. Otázkou však je,
% ktorej konvencii sa prispôsobiť.. týka sa to napríklad poľa
% citedate v OPmac-bib oproti poľu urldate v BibLaTeXu, ktoré
% zachytávajú presne ten istý údaj. Takisto aj formáty dátumu sa
% medzi týmito implementáciami líšia -- formát YYYY/MM/DD oproti
% YYYY-MM-DD (ISO 8601) -- čo spôsobí ignorovanie poľa na \uv{druhej
% strane}, pretože daná implementácia nebude schopná takéto pole
% rozparsovať a spracovať.


\section{Balík biblatex-iso690}

Spomedzi existujúcich riešení uvedených v~predošlej sekcii je pre
sadzbu v~\LaTeX{}u relevantný jedine balík \textsf{biblatex-iso690}.
Existujúce odchýlky od normy ISO 690 boli z~pôvodnej implementácie
odstránené a v~súčasnosti balík vyhovuje pravidlám poslednej verzie
normy ISO 690 pre tvorbu bibliografických odkazov a citácií.

Pôvodný stav balíka \textsf{biblatex-iso690} obsahoval nasledovné
chyby a nedostatky:

\begin{itemize}
\item pridržiavanie sa predošlých verzií noriem
  % edícia (nepovinné),
  % ' : ' (oddeľovač prvkov, nie interpunkčné znamienko),
  % anon (nemá sa používať termín "Anonym"),
  % miesto vydania + nakladateľ (nepovinné)
\item chybné poradie prvkov citácie
  % štandardné identifikátory,
  % číslovanie a stránkovanie
\item nadbytočná/chýbajúca interpunkcia
  % redundantná v prípade chýbajúcich údajov
\item chýbajúca podpora niektorých typov dokumentov % patent, thesis
\item chýbajúca podpora niektorých vyžadovaných prvkov citácie
  % edícia + číslo edície
\item chýbajúca sekundárna zodpovednosť % editori, prekladatelia
\item zastaraný kód % nedodržiavanie aktuálnych konvencií biblatexu
\end{itemize}

Pri revízii štýlu \textsf{biblatex-iso690} napokon došlo ku
kompletnej reimplementácii celého štýlu. Zmenou a korekciou prešli
nielen časti kódu, ktoré mali za účel vypisovať jednotlivé elementy
citovanej jednotky v správnom poradí. Boli to aj všetky pomocné
makrá, príkazy a definície zabezpečujúce korektné spracovávanie a
formátovanie bibliografických údajov (z bibliografickej databázy --
\texttt{.bib} súboru). Viaceré požiadavky normy bolo možné zabezpečiť
jednoducho priamo pomocou poskytovanej množiny funkcií a vďaka
možnostiam balíka BibLaTeX. Niektoré však bolo nutné vyriešiť
pozmenením základných makier a príkazov BibLaTeXu. Požiadavky,
ktoré nebolo možné zabezpečiť algoritmicky, príp. zostali ponechané
na tvorcu bibliografickej \texttt{.bib} databázy:

\begin{itemize}
\item chýbajúca podpora citovania formou priebežných poznámok
\item zalamovanie URL adries
\item algoritmické riešenie (ne)uvádzania prvého vydania publikácie
\item alg. riešenie (ne)uvádzania iba jedného (primárne prvého)
  vydavateľa
\item alg. riešenie (ne)uvádzania iba jedného (primárne prvého)
  miesta vydania
\item termín \textit{Anon} pre anonymné diela
\item lokalizačný reťazec \texttt{nodate} pre neznámy rok vydania
\end{itemize}

\subsection{Metódy citovania}

Norma ISO 690 predpisuje tri metódy citovania informačných zdrojov.
Okrem už spomenutej metódy formou priebežných poznámok je to tzv.
harvardská metóda (označovaná aj ako metóda autor-dátum) a tzv.
numerická metóda (forma číselného odkazu). V~balíku
\textsf{biblatex-iso690} sú tieto metódy dostupné ako samostatné
štýly pod názvom \texttt{iso-authoryear}, resp. \texttt{iso-numeric}.
Tieto štýly sa špecifikujú už v~preambule dokumentu pri zavádzaní
balíka BibLaTeX, napr.:

\cmdmint{\usepackage[style=iso-numeric]{biblatex}}

\subsection{Voľby balíka -- prispôsobiteľnosť}

Norma ISO 690 nepredpisuje pre bibliografické citácie konkrétny štýl,
formát a interpunkciu. Na zmenu výstupného formátu citácie sú
k~dispozícii voľby balíka \textsf{biblatex-iso690} a príp. je možné
aj predefinovanie v~preambule dokumentu.
Aktuálne dostupné voľby sú:
\begin{itemize}
\item \cmdmint{spacecolon=[true|false]} na zmenu výpisu dvojbodky
  pri názve a podnázve diela alebo pri nakladateľských informáciách
  \begin{itemize}
  \item Miesto : Nakladateľstvo
  \item Miesto: Nakladateľstvo
  \end{itemize}
\item \cmdmint{pagetotal=[true|false]} pre zobrazenie celkového
  počtu strán ako doplňujúcej informácie
  \begin{itemize}
  \item Miesto: Nakladateľstvo, 2008 [60 s.]
  \item Miesto: Nakladateľstvo, 2008
  \end{itemize}
\item \cmdmint{shortnumeration=[true|false]} pre typograficky
  odlíšený skrátený výpis informácií o~číslovaní
  \begin{itemize}
  \item \dots\ 2011, \textbf{32}(3), 289--301 [cit. 2016-05-14] \dots
  \item \dots\ 2011, roč. 32, č. 3, s. 289--301 [cit. 2016-05-14] \dots
  \end{itemize}
\item \cmdmint{thesisinfoinnotes=[true|false]} určenie poradia
  informácií o~záverečnej kvalifikačnej práci
  \begin{itemize}
  \item \dots\ Dostupné z: <\dots>. BP. MU, FI, Brno. Vedúci práce Petr SOJKA
  \item \dots\ BP. MU, FI, Brno. Vedúci práce Petr SOJKA. Dostupné z: <\dots>
  \end{itemize}
\end{itemize}

% priklady?

\subsection{Integrácia do šablóny fithesis3}

\textsf{Fithesis3} je oficiálna šablóna Masarykovej univerzity
na sadzbu záverečných kvalifikačných prác
v~\LaTeX{}u~\cite{novotny2015}. Trieda je navrhnutá s~možnosťou
jednoduchej rozšíriteľnosti i pre iné akademické inštitúcie.
Preto bolo prirodzenou požiadavkou začleniť aj balík
\textsf{biblatex-iso690} do štýlu \textsf{fithesis3} s~cieľom
maximalizovať používateľskú prívetivosť.
Integrácia do šablóny \textsf{fithesis3} prebiehala v~spolupráci
s~hlavným vývojárom tohoto balíka -- Vítom Novotným -- a spočíva
v~nasledovných krokoch:
\begin{itemize}
\item nový kľúč \texttt{bib} vo voľbách balíka \textsf{fithesis3},
  do ktorého sa špecifikujú bibliografické databázy (\texttt{.bib} súbory)
\item následné automatické zavedenie metódy citovania podľa zvolenej
  fakulty
\item automatická sadzba zoznamu bibliografických citácií na konci
  dokumentu
\item samozrejme je možná ručná špecifikácia vyššie uvedeného
  (viď.~\ref{biblatex:typesetting})
\end{itemize}

\begin{minted}[frame=lines]{latex}
\documentclass{fithesis3}
\thesissetup{
  ...
  bib = {<bibliografická-databáza.bib>}
  ...
}
\begin{document}
\cite{...}
...
\end{document}
\end{minted}

\subsection{Dostupnosť}

Ako už bolo spomenuté, pre systém \LaTeX{} doposiaľ neexistovala
žiadna oficiálna podpora tvorby bibliografických odkazov a citácií
podľa normy ISO 690. Až revíziou balíka \textsf{biblatex-iso690}
vznikla jeho oficiálna podoba, ktorá je teraz dostupná na
medzinárodnom archíve CTAN ako balík
\href{http://ctan.org/pkg/biblatex-iso690}{\textsf{biblatex-iso690}}.
Zároveň je pod rovnakým menom dostupný aj v~\TeX{}ovej distribúcii
\TeX{} Live 2016.


\section{Záver}

Článok sa zaoberá sadzbou bibliografie v~systéme \LaTeX{} podľa normy
ISO 690. Úvodom je načrtnutá problematika samotnej normy ISO 690 a
následne sú predstavené i možnosti sadzby bibliografie v~systéme
\LaTeX{}. Na základe existujúcich riešení s~podporou normy ISO 690
je bližšie predstavený balík \textsf{biblatex-iso690}, ktorý sa po
iniciálnej implementácii v~roku 2011 dočkal kompletnej
reimplementácie dodržujúc pravidlá poslednej revízie normy ISO 690.
Bibliografické citácie uvedené v~tomto článku sú vysádzané za
použitia implementovaného balíka \textsf{biblatex-iso690}, môžu teda
slúžiť ako referenčný \nameref{biblist}.

\printbibliography[title={Zoznam bibliografických citácií\label{biblist}}]

\end{refsection}

\end{document}
