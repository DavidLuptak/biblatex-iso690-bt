\PassOptionsToPackage{dvipsnames}{xcolor} % have more already defined colors
\documentclass{beamer}
\usetheme[
  university=mu,
  faculty=fi,
  logoLocale=czech,
]{fibeamer}

%\usecolortheme{crane}
\setbeamercovered{transparent} %% Watermarked items..

\AtBeginSection[] % Do nothing for \section*
{
\begin{frame}<beamer>
%\frametitle{Outline}
\tableofcontents[currentsection]
\end{frame}
}

\usepackage[utf8]{inputenc}
\usepackage[slovak]{babel}
\usepackage{amssymb}  %% Typesetting \checkmark
\usepackage{pifont}  %% Typesetting \xmark
\newcommand{\xmark}{\ding{55}}%
\usepackage{picture}  %% Typesetting curly bracket over two lines in itemize
\usepackage{minted}  %% Highlighted source code for LaTeX

\newcommand{\cmd}[1]{\mintinline{latex}{#1}}

\usepackage{csquotes}  %% Context sensitive quotation facilities
\usepackage[style=iso-numeric,shortnumeration]{biblatex}

\addbibresource{../resource.bib}
\addbibresource{reference-resource.bib}

\author{Dávid Lupták}
\title{Sadzba bibliografie podľa normy ISO 690}
\subtitle{Obhajoba bakalárskej práce}
% \date{2016-06-20}
% \institute[FI MU]{Fakulta informatiky\\Masarykova Univerzita, Brno}

%%%%%%%%%%%%%%%%%%%%%%%%%%%%%%%%%%
% biblatex environment adjustment
\DeclareFieldFormat{labelnumberwidth}{\mkbibbrackets{#1}}

\renewcommand\MethodFormat{%
  \printtext[labelnumberwidth]{%
    \printfield{labelprefix}%
    \printfield{labelnumber}}}%

\defbibenvironment{bibliography}
  {\list%
     {\MethodFormat}%
     {\setlength{\leftmargin}{7.5mm}%
      \setlength{\itemindent}{0mm}%
      \setlength{\itemsep}{\bibitemsep}%
      \setlength{\parsep}{\bibparsep}}%
      \renewcommand*{\makelabel}[1]{\hss##1}
      %\raggedright}
      }%
  {\endlist}%
  {\item}%
%%%%%%%%%%%%%%%%%%%%%%%%%%%%%%%%%%

% reset counter for every section
\makeatletter
\@addtoreset{footnote}{section}
\makeatother

\begin{document}

\frame{\titlepage}

\frame{\tableofcontents}

\section{Norma ISO 690}

% \begin{frame}{ČSN ISO 690:2011}
% \framesubtitle{Najnovšia verzia normy ISO 690}
% \fullcite{csn:iso690:2011}
% \end{frame}

% \begin{frame}{Terminológia\footnote[frame]{\fullcite{csn:iso690:2011}}}
% \begin{description}[citácia]
% \item[odkaz] údaj v~texte alebo iný druh obsahu dokumentu na príslušnú bibliografickú citáciu
% \item[citácia] dáta popisujúce informačný zdroj alebo jeho časť, dostatočne presne a podrobne na to, aby mohol byť tento zdroj identifikovaný a bolo možné ho vyhľadať
% \end{description}
% \end{frame}

\begin{refsection} % biblatex feature to separate bibliographies
\begin{frame}{Príklad na úvod}
\framesubtitle{Podľa najnovšej verzie normy ČSN ISO 690:2011}
\uv{\emph{[\dots] aj netriviálne deduktívne hry môžu byť vyriešené softvérovým nástrojom~\cite{Klimos2015minimal}, čo vedie k~záveru [\dots]}}
\rule{\textwidth}{0.2pt}
\printbibliography
\end{frame}
\end{refsection}

\begin{frame}{Zásada konzistencie}
Norma ISO 690
\begin{description}[nešpecifikuje]
\item[nešpecifikuje] konkrétny štýl bibliografického odkazu alebo citácie
\item[doporučuje] jednotný štýl, formát a interpunkciu
\end{description}
\begin{center}
$\implies$
\end{center}
\begin{description}[nešpecifikuje]
\item[uctieva] princíp oddelenia formy od obsahu
\item[nie je] citačný štýl
\end{description}
\end{frame}

\begin{frame}{Nejasné miesta a nepresnosti}
\framesubtitle{Viacjazyčnosť citácie}
\uv{\emph{\dots údaj by mal byť uvedený v~podobe, v~akej sa objavuje v~preferovanom prameni informácií citovaného dokumentu\dots}}
\newline
\begin{itemize}
\item číslovanie a stránkovanie
\item názov a číslo edície
\item etc.
\end{itemize}
\end{frame}

\section{Teoretická časť práce}

%\subsection{Existujúce riešenia}
\begin{frame}{Existujúce riešenia}
\framesubtitle{Generátory citácií, citačné manažéry, atď.}
\begin{itemize}
\item Citace.com
\item Zotero
\item JabRef
\item Citavi
\item OPmac-bib
\end{itemize}
\end{frame}

%\subsection{Sadzba bibliografie v~systéme \LaTeX}
\begin{frame}{Sadzba bibliografie v~systéme \LaTeX}
Tri základné prístupy:
\begin{enumerate}
\item \cmd{thebibliography}
\item \textsc{Bib}\TeX
\item BibLaTeX
      % thanks to the http://tex.stackexchange.com/a/12395
      \hspace{45pt}\makebox(0,0){\put(0,2.2\normalbaselineskip){%
        $\left.\rule{0pt}{1.1\normalbaselineskip}\right\}$ + \texttt{.bib} súbor}}
\end{enumerate}

\end{frame}

\section{Implementácia}

% \begin{frame}{Základná štruktúra BibLaTeXu}
% \begin{description}[.bbx]
% \item<1->[.bbx] súbory na definovanie bibliografických štýlov
% \item<2->[.cbx] súbory na definovanie citačných štýlov
% \item<3->[.dbx] súbory na definovanie nových dátových modelov
% \item<4->[.lbx] súbory na definovanie lokalizačných reťazcov
% \end{description}
% \end{frame}

\begin{frame}{Referenčná implementácia \textsf{biblatex-iso690}\footnote[frame]{\url{https://github.com/michal-h21/biblatex-iso690}}}
\framesubtitle{Autorom Michal Hoftich}
Pôvodný stav
\begin{itemize}
\item pridržiavanie sa predošlých verzií noriem % edícia (nepovinné), ' : ' (oddeľovač prvkov, nie interpunkčné znamienko), anon (nemá sa používať termín "Anonym"), miesto vydania + nakladateľ (nepovinné)
\item chybné poradie prvkov citácie % štandardné identifikátory, číslovanie a stránkovanie
\item (nadbytočná/chýbajúca) interpunkcia % redundantná v prípade chýbajúcich údajov
\item chýbajúca podpora niektorých typov dokumentov % patent, thesis
\item chýbajúca podpora niektorých vyžadovaných prvkov citácie % edícia + číslo edície
\item chýbajúca sekundárna zodpovednosť % editori, prekladatelia
\item zastaraný kód % nedodržiavanie aktuálnych konvencií biblatexu
\end{itemize}
\end{frame}

\begin{frame}{Ukážka}
\framesubtitle{Citácia vygenerovaná pôvodnou verziou}
\begin{thebibliography}{9}
\bibitem{t03} % Lecture Notes in Computer Science
  KLIMOŠ, Miroslav; KUČERA, Antonín. Cobra: A~Tool for Solving General Deductive Games. In \emph{Proceedings of 20th International Conference on Logic for Programming, Artificial Intelligence and Reasoning (LPAR 2015)} [online]. \textcolor<2>{Red}{Č. 9450}. Heidelberg: Springer, 2015 [cit. 2016-06-15], s. 31–47. \textcolor<2>{Red}{LNCS}. \textcolor<2>{Dandelion}{Url: <\url{http://dx.doi.org/10.1007/978-3-662-48899-7_3}>}. \textcolor<2>{Dandelion}{ISBN 978-3-662-48899-7}.
\end{thebibliography}
\end{frame}

\begin{frame}{Aktuálny stav}
\framesubtitle{Balík \textsf{biblatex-iso690} [verzia 0.3.1]}
\begin{itemize}
\item korektné spracovávanie, formátovanie a výpis údajov
  \begin{itemize}
  \item makrá
  \item príkazy
  \item definície
  \end{itemize}
\item na úrovni BibLaTeXu alebo \TeX u
\item nové typy a polia záznamov
\item nové lokalizačné reťazce
\end{itemize}
\ $\implies$ dodržiavanie aktuálnej verzie normy
\end{frame}

\begin{frame}{Rôzne metódy citovania}
\framesubtitle{\dots a predsa vychádzajú z~toho istého kódu}
\textbf{autor-dátum: autor aj editor}\\
\uv{\emph{[\dots] príslušný odkaz v~texte na toto dielo~(\textcolor{Dandelion}{Kováč}, 2016) [\dots]}}\\
KOVÁČ, Peter, \textcolor{Green}{2016}. \emph{Veľdielo}. \textcolor{Red}{Ed. NOVÁK, Ján}. Miesto: Nakladateľstvo

\rule{\textwidth}{0.2pt}

\textbf{autor-dátum: iba editor}\\
\uv{\emph{[\dots] príslušný odkaz v~texte na toto dielo~(\textcolor{Dandelion}{Novák}, 2016) [\dots]}}\\
\textcolor{Red}{NOVÁK, Ján (ed.)}, \textcolor{Green}{2016}. \emph{Veľdielo}. Miesto: Nakladateľstvo

\rule{\textwidth}{0.2pt}

\textbf{číslované odkazy}\\
\uv{\emph{[\dots] príslušný odkaz v~texte na toto dielo~[1] [\dots]}}\\
KOVÁČ, Peter. \emph{Veľdielo}. Ed. NOVÁK, Ján. Miesto: Nakladateľstvo, \textcolor{Green}{2016}
\end{frame}

% \begin{refsection} % biblatex feature to separate bibliographies
% \begin{frame}{Jazyk CSL vs. BibLaTeX}
% % csl numeric-slovak
% \begin{thebibliography}{1}
% \bibitem{Priezvisko2016}
% PRIEZVISKO \textit{Názov: podnázov}. . revidované vydanie. vyd. Miesto: Nakladateľstvo, 2016. .
% \end{thebibliography}

% \nocite{Priezvisko2016}
% \printbibliography
% \end{frame}
% \end{refsection}

\section{Záver}

\begin{frame}{Výsledky}
\begin{itemize}
\item kompletná reimplementácia referenčnej implementácie balíka \textsf{biblatex-iso690}
\item dokumentácia v~českom i anglickom jazyku
\item oficiálne začlenenie a vydanie štýlu
  \begin{itemize}
  \item do šablóny na sadzbu záverečných kvalifikačných prác \textsf{fithesis3}\footnote[frame]{\url{https://www.ctan.org/pkg/fithesis}}
  \item na medzinárodný archív CTAN\footnote[frame]{\url{https://www.ctan.org/pkg/biblatex-iso690}}
    \begin{itemize}
    \item súčasťou \TeX\ Live 2016
    \end{itemize}
  \end{itemize}
\item rozšírenie balíka \textsf{biblatex}\footnote[frame]{\url{https://www.ctan.org/pkg/biblatex}} o~slovenskú lokalizáciu
\end{itemize}
\end{frame}

%%%%%%%%%%%%%
%%%% Q&A %%%%
%%%%%%%%%%%%%

\begin{darkframes}
\begin{frame}[plain]{%
  \mbox{}
  \newcounter{qaFrameNumber}
  \setcounter{qaFrameNumber}{\value{framenumber}}
  \author{}
  \title{Námety k~diskusii}
  \subtitle{}
  \titlepage
  \setcounter{framenumber}{\value{qaFrameNumber}}
}
\end{frame}
\end{darkframes}

%% OPONENT %%
%%%%%%%%%%%%%

%% Q: V práci proto mohlo být do větší hloubky popsáno, jaké změny bylo třeba udělat pro převedení existujícího kódu do souladu s poslední verzí normy, jaký objem práce to představovalo..
%% A: Pri tvorbe štýlu biblatex-iso690 napokon došlo ku kompletnej reimplementácii celého štýlu. Zmenou a korekciou prešli nielen časti kódu, ktoré mali za účel vypisovať jednotlivé elementy citovanej jednotky v správnom poradí. Boli to aj všetky pomocné makrá, príkazy a definície zabezpečujúce korektné spracovávanie a formátovanie bibliografických údajov (z bibliografickej databázy -- .bib súboru). Viaceré požiadavky normy bolo možné zabezpečiť jednoducho priamo pomocou poskytovanej množiny funkcií a vďaka možnostiam balíka BibLaTeX. Niektoré však bolo nutné vyriešiť pozmenením základných makier a príkazov BibLaTeXu. Požiadavky, ktoré nebolo možné zabezpečiť algoritmicky, sú popísané v dokumentácii balíka.
%%
%% + vznik anglickej dokumentácie

%% Q: ..a co bylo nutné udělat pro integraci se šablonou fithesis3.
%% A: Vo všeobecnosti spadá použitie štýlu pod načítanie balíka biblatex v preambuli dokumentu s voľbou metódy citovania špecifikovanej balíkom biblatex-iso690. To znamená, že je samozrejme nutná prítomnosť balíka biblatex-iso690 na stroji, na ktorom prebieha preklad dokumentu, avšak tento balík stavia na balíku biblatex a preto je použité práve \usepackage[style=<biblatex-iso690 citestyle method>]{biblatex} v preambuli dokumentu.
%%    Zahrnutie do šablóny fithesis3 spočíva v nasledovných krokoch:
%%    * nový kľúč `bib` vo voľbách balíka fithesis3, do ktorého sa špecifikujú bibliografické databázy (.bib súbory)
%%    * následné automatické zavedenie metódy citovania podľa zvolenej fakulty
%%    * automatická sadzba zoznamu bibliografických citácií na konci dokumentu
%%    * samozrejme je možná ručná špecifikácia vyššie uvedeného
%%    Integrácia do balíka fithesis3 prebiehala v spolupráci s hlavným vývojárom tohoto balíka -- Vítom Novotným.

%% Q: Podrobněji však mohlo být popsáno, co se případně dle požadavků normy implementovat nepodařilo.
%% A: Áno, je pravda že v práci nie je explicitne uvedený zoznam všetkých `nepodarkov`. Nasleduje pokus o kompletný zoznam:
%%    * chýbajúca podpora citovania formou priebežných poznámok
%%    * zalamovanie url adries
%%    * algoritmické riešenie (ne)uvádzania prvého vydania publikácie
%%    * algoritmické riešenie (ne)uvádzanie iba jedného (primárne prvého) vydavateľa
%%    * algoritmické riešenie (ne)uvádzanie iba jedného (primárne prvého) miesta vydania
%%    * `Anon` pre anonymné diela
%%    * `nodate` pre neznámy rok vydania
%%    * možné problémy s nejakými špecifickými typmi citovaných jednotiek

%% Q: V repositáři upstream verze projektu na GitHub je však v tomto okamžiku otevřeno 16 záznamů v issue trackeru. Nejsou minimálně některé z nich relevantní pro shodu s normou?
%% A: Áno, väčšina z nich je, a aj sa z nich čerpalo. Pri mnohých commitoch je možné nájsť <stereotype> `Addressing #n`, čo sa vzťahuje práve na príslušné issues alebo diskusie v rámci pull requestov. Tickety však akurát neboli uzavreté.

%% Q: Vzhledem k zaměření práce mi mírně vadilo uvádění odkazů do seznamu literatury v textu práce.
%% A: V istom zdroji som našiel, že uvádzanie odkazov až za celou, bodkou ukončenou, vetou je (typografická) chyba. Tento zdroj však momentálne neviem dohľadať. Možno totiž ani nesúvisel priamo s normou ISO 690, skôr obecne s metódami citovania.

%% Q: Na straně 17 by bylo vhodné k příkladu nahoře doplnit ukázku výstup..
%% A: Áno, určite by to pochopeniu obsahu textu len prospelo.

%% Q: ..na str. 18 chybí vysvětlení významu parametrů příkazu Authors(...).
%% A: Spomenuté argumenty sú asociované so zoznamom možností formátovania v danej sekcii, naozaj by však bolo žiadúce rozpísať sa pri tomto ilustračnom príklade viac.

%% Q: Kam se poděla první chyba z úvodu sekce 3.3.9 ze str. 19?
%% A: Bola zakomentovaná a nasledujúci odstavec nebol voči tejto skutočnosti adekvátne upravený.

%% Q: Správné jméno souboru na str. 36 je biblatex_.def, nikoliv biblatex$_$.def.
%% A: Pozostatok nepresnosti v dokumentácii balíka biblatex.

\begin{frame}[fragile]
\emph{Je v~rámci \textsf{biblatex-iso690} nějak řešeno verzování \uv{nekompatibilních} změn výstupního formátu stylu na uživatelské úrovni (volby \cmd{\usepackage} apod.) [\dots]?} %, aby se dalo zabránit změně vzhledu dokumentu při jeho přesazbě novější verzí balíku biblatex-iso690?}
\medskip\par
% Obohatenie dokumentácie o sekciu prispôsobovania štýlu, formátu a interpunkcie na TODO liste
Norma nepredpisuje konkrétny štýl, formát a interpunkciu, balík BibLaTeX ponúka veľkú flexibilitu, preto:
\begin{itemize}
\item predefinovanie v~preambule dokumentu
  \begin{itemize}
  \item \cmd{\renewcommand\multinamedelim{\addcomma\space}}
    \begin{itemize}
    \item PRIEZVISKO, Meno\underline{, }PRIEZVISKO, Meno
    \end{itemize}
  \end{itemize}
\item voľby balíka \textsf{biblatex-iso690}, príp. \textsf{biblatex}
  \begin{itemize}
  \item \cmd{shortnumeration=[true|false]}
    \begin{itemize}
    \item \dots\ 2011, \textbf{32}(3), 289--301 [cit. 2016-05-14] \dots
    \item \dots\ 2011, roč. 32, č.3, s. 289--301 [cit. 2016-05-14] \dots
    \end{itemize}
  \end{itemize}
\end{itemize}
\end{frame}

\begin{frame}
\emph{Jaká byla kritéria pro výběr nástrojů revidovaných v~kapitole 3?}
\medskip\par
\begin{itemize}
\item dostupnosť zdrojových kódov jednotlivých implementácií
\item v~prípade Citace.com popularita tohoto nástroja v~českom akademickom prostredí
\item poukázanie na problematiku výkladu normy
\item porovnanie a použiteľnosť s~BibLaTeXom
\end{itemize}
\end{frame}

\begin{frame}
\emph{Na stranách 9--10 v~kapitole 3.1.7 je v~práci popisován ISO 690 výstup nástroje Citace.com. Je uvedený výčet zjištěných odchylek od normy kompletní? [\dots]} %Jiné odchylky zjištěny nebyly, nebo jen nebyly uvedeny v~textu práce? Odchylky byly zjištěny vlastním zkoumáním, anebo jsou již někde dokumentovány?}
\medskip\par
\begin{itemize}
\item zoznam nie je kompletný -- odchýlky sa prelínajú aj skrz ostatné popisované nástroje
\item odchýlky nebolo možné zistiť všetky, nakoľko nie sú k~dispozícii zdrojové súbory
\item vlastné skúmanie -- online generátor citácií
\end{itemize}
\end{frame}

\begin{frame}[fragile]
\emph{Proč byla do archivu v~IS MU vložena verze odlišná (jednostranný tisk) od výtisku
pro oponenta (oboustranný tisk)? [\dots]}
\medskip\par
% just the misunderstanding of the documentation
% https://github.com/Witiko/fithesis3/commit/caa751148f2785f909da4f75be50bd7b7cbb8882
\begin{minted}[fontsize=\scriptsize]{diff}
diff --git a/guide/mu/guide.dtx b/guide/mu/guide.dtx
index be770b3..edf0fd5 100644
--- a/guide/mu/guide.dtx
+++ b/guide/mu/guide.dtx
@@ -410,6 +410,13 @@
+
+      \emph{You may feel compelled to use \thguide@itemfmt{twoside}
+      for the printed version of the thesis and \thguide@itemfmt{%
+      oneside} for the digital version to reduce the number of
+      blank pages. This will, however, cause the page numbers to
+      differ between the printed and the digital version, which
+      will make it difficult to cite your work. Do not do it.}
\end{minted}
\url{https://github.com/Witiko/fithesis3/commit/caa751148f2785f909da4f75be50bd7b7cbb8882}
\end{frame}

%% VEDÚCI %%
%%%%%%%%%%%%

%% Q: Některá slova šla dělit lépe (bibli-ografické str. 36, zau-žívané str. 37).
%% A: Áno, do vzorov delenia ani úprav typografických rozšírení nebolo zasahované.

\begin{frame}
\emph{Jakou zpětnou vazbu od uživatelů jste získal?}
\medskip\par
% Spätná väzba bola doposiaľ prijatá len od pár známych, ktorí tento semester tiež robili bakalársku prácu.
\begin{itemize}
\item[\checkmark] inštalácia BibLaTeXu a balíka \textsf{biblatex-iso690}
\item[\checkmark] preklad dokumentu (rozdiel medzi \textsf{biblatexom} a \textsf{biberom})
\item[\checkmark] chýbajúca slovenská lokalizácia
\item[\checkmark] formát dátumu ISO 8601 (tj. yyyy-mm-dd)
\item[\checkmark] formát \emph{dostupné z~WWW: <link>}
\item[\checkmark] požiadavka na začlenenie do \textsf{fithesis3}
% \item prostredníctvom príspevku na diskusnom fóre nič
\end{itemize}
\end{frame}

\begin{frame}
\emph{Je styl použitelný i v~jiných jazykových prostředích a s~jakými omezeními?}
\medskip\par
\begin{itemize}
\item balíky \textsf{babel}/\textsf{polyglossia} voliteľné (avšak doporučované) pre použitie s~\textsf{biblatexom}
  \begin{itemize}
  \item \checkmark{} \textsf{babel} a \xmark{} \textsf{biblatex}
    \begin{itemize}
    \item \textbf{urlalso}
    \end{itemize}
  \item \xmark{} \textsf{babel} a \xmark{} \textsf{biblatex}
    \begin{itemize}
    \item Available also from
    \end{itemize}
  \end{itemize}
\item pull request nemeckej lokalizácie\footnote[frame]{\url{https://github.com/michal-h21/biblatex-iso690/pull/48}}
\end{itemize}
\end{frame}

\begin{frame}[fragile]
\emph{V~práci zmiňujete nedostatky řešení ISO 690 citací v~ISu. Jaké to jsou a jak se dají napravit?}
\medskip\par
\begin{itemize}
\item obmedzené možnosti vytvárania publikácií -- nemožné vytvoriť záznam typu \uv{iné}
\item export do \texttt{.bib} súboru
\end{itemize}

\begin{minted}[fontsize=\scriptsize]{tex}
@misc{Luptákthesis,
AUTHOR = "LUPTÁK, Dávid",
TITLE = "Sazba bibliografie dle normy ISO 690 [online]",
YEAR = " [cit. 2016-06-14]",
TYPE = "Bakalářská práce", 
SCHOOL = "Masarykova univerzita, Fakulta informatiky, Brno",
SUPERVISOR = "Petr Sojka", 
URL = "Dostupné z~WWW <http://is.muni.cz/th/422640/fi_b/>",
}
\end{minted}
\end{frame}

\begin{frame}
Typ \cmd{@misc}:
\newline
\fullcite{LuptakMisc}
\newline
\newline
Typ \cmd{@thesis}:
\newline
\fullcite{LuptakThesis}
\end{frame}

\begin{frame}[fragile]
Zdroj:
\begin{minted}[fontsize=\scriptsize]{tex}
@thesis{LuptakThesis,
AUTHOR = "Lupták, Dávid",
TITLE = "Sazba bibliografie dle normy ISO 690",
HOWPUBLISHED = "online",
URLDATE = "2016-06-14",
TYPE = "Bakalářská práce", 
SCHOOL = "Masarykova univerzita, Fakulta informatiky, Brno",
SUPERVISOR = "Petr Sojka", 
URL = "http://is.muni.cz/th/422640/fi_b/",
}
\end{minted}

Výstup:
\newline
\fullcite{Luptak2016}
\end{frame}

\begin{frame}
\emph{Nestojí za to výsledky publikovat například aspoň v~Zpravodaji CSTUGu?}
\medskip\par
Určite áno a nielen v~Spravodaji\dots\\
\dots existuje niekoľko relevantných kurzov aj priamo na MU:
\begin{description}[FF:PLIN028]
\item[FI:PB029] Elektronická příprava dokumentů
%% DONE
\item[FI:VB000] Základy odborného stylu
%% sylabus jaro 2015: ČSN ISO 690 a ČSN ISO 690-2, odkazy na predchádzajúce verzie noriem/interpretácií
%% https://is.muni.cz/auth/el/1433/jaro2015/VB000/
%% THE FIRST BIGGEST FAIL
%% citace                 <> odkaz
%% bibliografická citace  <> citácia
%% pozostatok z predošlých verzií noriem
\item[PřF:M5751] Elektronická sazba a publikování v~\TeX u
%% referencia na "neoficiálny" štýl od Michala Hofticha
\item[FF:PLIN028] Počítačová sazba textových dokumentů
%% využívajú BibTeX
\end{description}
\end{frame}

\begin{darkframes}
\begin{frame}[plain]{%
  \mbox{}
  \newcounter{thanksFrameNumber}
  \setcounter{thanksFrameNumber}{\value{framenumber}}
  \title{Ďakujem za pozornosť!}
  \subtitle{}
  \titlepage
  \setcounter{framenumber}{\value{thanksFrameNumber}}
}
\end{frame}
\end{darkframes}

\end{document}

% simple _template_ for q&a
\begin{frame}{}
\emph{}
\medskip\par

\end{frame}
