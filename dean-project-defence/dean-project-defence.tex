\PassOptionsToPackage{dvipsnames}{xcolor} % have more already defined colors
\documentclass{beamer}
\usetheme[
  university=mu,
  faculty=fi,
  logoLocale=czech,
]{fibeamer}

%\usecolortheme{crane}
\setbeamercovered{transparent} %% Watermarked items..

\AtBeginSection[] % Do nothing for \section*
{
\begin{frame}<beamer>
%\frametitle{Outline}
\tableofcontents[currentsection]
\end{frame}
}

\usepackage[utf8]{inputenc}
\usepackage[slovak]{babel}
\usepackage{amssymb}  %% Typesetting \checkmark
\usepackage{minted}  %% Highlighted source code for LaTeX

\newcommand{\cmd}[1]{\mintinline{latex}{#1}}

\usepackage{csquotes}  %% Context sensitive quotation facilities
\usepackage[style=iso-numeric,shortnumeration]{biblatex}

\addbibresource{reference-resource.bib}

% Demo showing how the iso style could be easily switched
% \url{https://www.overleaf.com/6955878njcdygbdnggw}

\author{Dávid Lupták}
\title{ČSN ISO 690:2011 styl pro BibLaTeX}
\subtitle{Obhajoba projektu}
% \date{2016-11-10}
% \institute[FI MU]{Fakulta informatiky\\Masarykova Univerzita, Brno}

% reset counter for every section
\makeatletter
\@addtoreset{footnote}{section}
\makeatother

\begin{document}

\frame{\titlepage}

\begin{frame}{Anotácia projektu}
Cílem projektu je připravit bibliografický styl pro balík BibLaTeX, který bude odpovídat nové citační normě ČSN ISO 690:2011. Bibliografický styl bude začleněn do balíku fithesis3 pro sazbu závěrečných prací na Masarykově univerzitě, který se tak stane snáze použitelným na fakultách a ústavech, které dodržování normy při přípravě závěrečných prací vyžadují.
\end{frame}

\frame{\tableofcontents}

\section{Norma ISO 690}

\begin{frame}{Čo sa zmenilo v~norme ISO 690}
\framesubtitle{Oproti predchádzajúcim verziám}
\[
\underbrace{\phantom{\text{-2}}\strut\text{\Large{ISO 690:1987 + ISO 690-2:1997}}}_\text{}
\]
\begin{center}
\LARGE{ISO 690:2010}
\end{center}
\begin{center}
\textcolor{Gray}{ČSN ISO 690:2011, STN ISO 690:2012, DIN ISO 690:2013, etc.}
\end{center}
\end{frame}

\begin{frame}{Čo sa zmenilo v~norme ISO 690}
\framesubtitle{Oproti predchádzajúcim verziám}
\begin{itemize}
  \item Nové druhy informačných zdrojov
    \begin{itemize}
    \item kartografické materiály
    \item audiovizuálne diela
    \item etc.
    \end{itemize}
  \item Explicitné zdelenie, že norma nepredpisuje konkrétny štýl odkazu alebo citácie
  \item Chýbajúce termíny, napr. \emph{kapitola}
  \item Podrobnejší výklad pravidiel tvorby jednotlivých bibl. údajov
    \begin{itemize}
      \item zmenené alebo doplnené
    \end{itemize}
\end{itemize}
\end{frame}

\section{Balík biblatex-iso690}

\begin{frame}{Referenčná implementácia \textsf{biblatex-iso690}\footnote[frame]{\url{https://github.com/michal-h21/biblatex-iso690}}}
\framesubtitle{Autorom Michal Hoftich}
Pôvodný stav
\begin{itemize}
\item pridržiavanie sa predošlých verzií noriem % edícia (nepovinné), ' : ' (oddeľovač prvkov, nie interpunkčné znamienko), anon (nemá sa používať termín "Anonym"), miesto vydania + nakladateľ (nepovinné)
\item chybné poradie prvkov citácie % štandardné identifikátory, číslovanie a stránkovanie
\item (nadbytočná/chýbajúca) interpunkcia % redundantná v prípade chýbajúcich údajov
\item chýbajúca podpora niektorých typov dokumentov % patent, thesis
\item chýbajúca podpora niektorých vyžadovaných prvkov citácie % edícia + číslo edície
\item chýbajúca sekundárna zodpovednosť % editori, prekladatelia
\item zastaraný kód % nedodržiavanie aktuálnych konvencií biblatexu
\end{itemize}
\end{frame}

\begin{frame}{Kompletná reimplementácia}
\framesubtitle{Balíka \textsf{biblatex-iso690}}
Makrá, príkazy a definície
\begin{itemize}
\item výpis jednotlivých elementov v~správnom poradí
\item spracovávanie a formátovanie bibliografických údajov
\item viacjazyčnosť citácie
\item ďalej prispôsobiteľné
\end{itemize}

Z~užívateľského pohľadu
\begin{itemize}
\item nové typy a polia záznamov
\item nové lokalizačné reťazce
\item nové voľby balíka na zmenu výstupného formátu
\end{itemize}
\end{frame}

\begin{frame}{Integrácia do balíka fithesis3}
\framesubtitle{V~spolupráci s~hlavným vývojárom tohoto balíka -- Vítom Novotným}
Zahrnutie do šablóny fithesis3 spočíva v~nasledovných krokoch
\begin{itemize}
\item nový kľúč `bib` vo voľbách balíka fithesis3, do ktorého sa špecifikujú bibliografické databázy (.bib súbory)
\item následné automatické zavedenie metódy citovania podľa zvolenej fakulty
\item automatická sadzba zoznamu bibliografických citácií na konci dokumentu
\end{itemize}

Pozn.: samozrejme je možná ručná špecifikácia vyššie uvedeného
\end{frame}

\begin{frame}[fragile]{MWE pre fithesis3}
\begin{minted}{latex}
\documentclass{fithesis3}
\thesissetup{
  ...
  bib = {<bibliografická-databáza.bib>}
  ...
}
...
\begin{document}
\cite{...}
...
\end{document}
\end{minted}
\end{frame}

\begin{frame}[fragile]{MWE pre biblatex-iso690 obecne}
\begin{minted}{latex}
\documentclass{...}
\usepackage[style=iso-authoryear]{biblatex}
\addbibresource{<bibliografická-databáza.bib>}
...
\begin{document}
\parencite{...}
\parencite*{...}
...
\printbibliography
\end{document}
\end{minted}
\end{frame}

\section{Príklady}

\begin{frame}{Ukážka}
\framesubtitle{Citácia vygenerovaná pôvodnou a aktuálnou verziou}
\dots Názov~[online]. \textcolor{Red}{Č. edície.} Miesto: Nakladateľstvo, 2016 [cit. 2016-11-08], s. 31--47. \textcolor{Red}{Edícia.} Url: <cesta>. ISBN
\[
\Downarrow
\]
\dots Názov~[online]. Miesto: Nakladateľstvo, 2016, s. 31--47 [cit. 2016-11-08]. \textcolor{Green}{Edícia, č. edície.} ISBN. Dostupné z~<url>
\end{frame}

\begin{frame}{Ukážka}
\framesubtitle{Rôzne metódy citovania a absencia údajov}
\textbf{autor-dátum: autor aj editor}\\
\uv{\emph{[\dots] príslušný odkaz v~texte na toto dielo~(\textcolor{Dandelion}{Kováč}, 2016) [\dots]}}\\
KOVÁČ, Peter, \textcolor{Green}{2016}. \emph{Veľdielo}. \textcolor{Red}{Ed. NOVÁK, Ján}. Miesto: Nakladateľstvo

\rule{\textwidth}{0.2pt}

\textbf{autor-dátum: iba editor}\\
\uv{\emph{[\dots] príslušný odkaz v~texte na toto dielo~(\textcolor{Dandelion}{Novák}, 2016) [\dots]}}\\
\textcolor{Red}{NOVÁK, Ján (ed.)}, \textcolor{Green}{2016}. \emph{Veľdielo}. Miesto: Nakladateľstvo

\rule{\textwidth}{0.2pt}

\textbf{číslované odkazy}\\
\uv{\emph{[\dots] príslušný odkaz v~texte na toto dielo~[1] [\dots]}}\\
KOVÁČ, Peter. \emph{Veľdielo}. Ed. NOVÁK, Ján. Miesto: Nakladateľstvo, \textcolor{Green}{2016}
\end{frame}

\begin{frame}[fragile]{Ukážky vybraných typov záznamov}
\fullcite{Ryan1998}
\newline
\begin{minted}[fontsize=\scriptsize]{latex}
@misc{Ryan1998,
  title = {Saving Private Ryan},
  howpublished = {DVD},
  editora = {Steven Spielberg},
  editoratype = {Réžia},
  publisher = {Paramount},
  date = {1998}
}
\end{minted}
\end{frame}

\begin{frame}[fragile]
\fullcite{LibreOfficeLatest}
\newline
\begin{minted}[fontsize=\scriptsize]{latex}
@misc{LibreOfficeLatest,
  author = {Ruediger Timm and others},
  title = {Libre Office},
  howpublished = {softvér},
  publisher = {The Document Foundation},
  date = {2000/2016},
  version = {5.2.3},
  url = {https://www.libreoffice.org/download/libreoffice-fresh},
  note = {Požiadavky na systém: Linux kernel verzie 2.6.18 alebo vyššej...}
}
\end{minted}
\end{frame}

\begin{frame}[fragile]
\fullcite{Groll2008}
\newline
\begin{minted}[fontsize=\scriptsize]{latex}
@patent{Groll2008,
  author = {Clad Metals LLC Canonsburg, PA 15317 (US)},
  title = {Method of making a copper core five-ply composite and...},
  editora = {Groll, W.A.},
  editoratype = {Vynálezca:},
  publisher = {Google Patents},
  classification = {EP 1 094 937 B1},
  type = {patenteu},
  url = {https://encrypted.google.com/patents/EP1094937B1}
}
\end{minted}
\end{frame}

\begin{frame}[fragile]
\fullcite{Luptak2016}
\newline
\begin{minted}[fontsize=\scriptsize]{tex}
@thesis{Luptak2016,
AUTHOR = "Lupták, Dávid",
TITLE = "Sazba bibliografie dle normy ISO 690",
HOWPUBLISHED = "online",
URLDATE = "2016-06-14",
TYPE = "Bakalářská práce",
SCHOOL = "Masarykova univerzita, Fakulta informatiky, Brno",
SUPERVISOR = "Petr Sojka",
URL = "http://is.muni.cz/th/422640/fi_b/",
}
\end{minted}
\end{frame}

\section{Záver}

\begin{frame}{Výsledky}
\begin{itemize}
\item[\checkmark] kompletná reimplementácia referenčnej implementácie balíka \textsf{biblatex-iso690}
\item[\checkmark] spĺňanie pravidiel normy ISO 690
\item[\checkmark] dokumentácia v~českom i anglickom jazyku
  \begin{itemize}
    \item ukážky použitia
    \item rady a varovania
  \end{itemize}
\item[\checkmark] oficiálne začlenenie a vydanie štýlu
  \begin{itemize}
  \item do šablóny na sadzbu záverečných kvalifikačných prác \textsf{fithesis3}\footnote[frame]{\url{https://www.ctan.org/pkg/fithesis}}
  \item na medzinárodný archív CTAN\footnote[frame]{\url{https://www.ctan.org/pkg/biblatex-iso690}}
    \begin{itemize}
    \item súčasťou \TeX\ Live 2016
    \end{itemize}
  \end{itemize}
\end{itemize}
\end{frame}

\begin{frame}{Výsledky}
\begin{itemize}
  \item[\checkmark] rozšírenie balíka \textsf{biblatex}\footnote[frame]{\url{https://www.ctan.org/pkg/biblatex}} o~slovenskú lokalizáciu
  \item[\checkmark] pribúdajúce pull requesty od komunity (napr. jazykové lokalizácie)
  \item[\checkmark] 47 označení \uv{obľúbené} projektu na GitHub repozitári
  \item[\checkmark] 17 forkov projektu na GitHub repozitári
\end{itemize}
\end{frame}


\begin{darkframes}
\begin{frame}[plain]{%
  \mbox{}
  \newcounter{thanksFrameNumber}
  \setcounter{thanksFrameNumber}{\value{framenumber}}
  \title{Ďakujem za pozornosť!}
  \subtitle{}
  \titlepage
  \setcounter{framenumber}{\value{thanksFrameNumber}}
}
\end{frame}
\end{darkframes}

\end{document}

% simple _template_ for q&a
\begin{frame}{}
\emph{}
\medskip\par

\end{frame}
