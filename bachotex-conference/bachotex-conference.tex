\PassOptionsToPackage{dvipsnames}{xcolor} % have more already defined colors
\documentclass{beamer}
\usetheme[
  university=mu,
  faculty=fi,
  logoLocale=english,
]{fibeamer}

%\usecolortheme{crane}
\setbeamercovered{transparent} %% Watermarked items..

\AtBeginSection[] % Do nothing for \section*
{
\begin{frame}<beamer>
%\frametitle{Outline}
\tableofcontents[currentsection]
\end{frame}
}

\usepackage[utf8]{inputenc}
\usepackage[slovak,english]{babel}
\usepackage{amssymb}  %% Typesetting \checkmark
\usepackage{minted}  %% Highlighted source code for LaTeX
\usepackage{tabularx}  %% Tabulars with adjustable-width columns
\usepackage{booktabs}  %% Publication quality tables in LaTeX -- \toprule, etc.

\newcommand{\cmd}[1]{\mintinline{latex}{#1}}

\usepackage{csquotes}  %% Context sensitive quotation facilities
\usepackage[style=iso-numeric,shortnumeration,pagetotal,thesisinfoinnotes=false]{biblatex}

\addbibresource{reference_resource.bib}

% Demo showing how the iso style could be easily switched
% \url{https://www.overleaf.com/6955878njcdygbdnggw}

\author{Dávid Lupták}
\title{Typesetting Bibliographies Compliant with the Standard ISO 690 in LaTeX}
\subtitle{BachoTeX 2017}
% \date{2016-12-17}
% \institute[FI MU]{Fakulta informatiky\\Masarykova Univerzita, Brno}

%%%%%%%%%%%%%%%%%%%%%%%%%%%%%%%%%%
% biblatex environment adjustment
\DeclareFieldFormat{labelnumberwidth}{\mkbibbrackets{#1}}

\renewcommand\MethodFormat{%
  \printtext[labelnumberwidth]{%
    \printfield{labelprefix}%
    \printfield{labelnumber}}}%

\defbibenvironment{bibliography}
  {\list%
     {\MethodFormat}%
     {\setlength{\leftmargin}{7.5mm}%
      \setlength{\itemindent}{0mm}%
      \setlength{\itemsep}{\bibitemsep}%
      \setlength{\parsep}{\bibparsep}}%
      \renewcommand*{\makelabel}[1]{\hss##1}
      %\raggedright}
      }%
  {\endlist}%
  {\item}%
%%%%%%%%%%%%%%%%%%%%%%%%%%%%%%%%%%

% reset counter for every section
\makeatletter
\@addtoreset{footnote}{section}
\makeatother

\begin{document}

\frame{\titlepage}
\frame{\tableofcontents}

\section{ISO 690 Standard}

\begin{frame}{ISO 690 Standard}
\begin{itemize}
\item Information and documentation -- Guidelines for bibliographic
      references and citations to information resources
\item Published by International Organization for Standardization (ISO)
\item One of the most frequent guidelines for citations and references
\end{itemize}
\end{frame}

\begin{frame}{ISO 690 Editions}
\framesubtitle{History}
\[
\underbrace{\phantom{\text{-2}}\strut\text{\Large{ISO 690:1987 + ISO 690-2:1997}}}_\text{}
\]
\begin{center}
\LARGE{ISO 690:2010}
\end{center}
\begin{center}
\textcolor{Gray}{ČSN ISO 690:2011, STN ISO 690:2012, DIN ISO 690:2013, etc.}
\end{center}
\end{frame}

\begin{frame}{Terminology}
\framesubtitle{Defined in the most recent edition of ISO 690:2010}
\begin{description}[reference]
\item[citation] indication within the text or other form of content
      of a relevant reference
\item[reference] data describing a resource or part thereof,
      sufficiently precise and detailed to identify it and to enable
      it to be located
\end{description}
\end{frame}

\begin{refsection} % biblatex feature to separate bibliographies
\begin{frame}{Terminology by example}
\framesubtitle{Defined in the most recent edition of ISO 690:2010}
\uv{\emph{[\dots] aj netriviálne deduktívne hry môžu byť vyriešené softvérovým nástrojom~\cite{Klimos2015minimal}, čo vedie k~záveru [\dots]}}
\rule{\textwidth}{0.2pt}
\printbibliography
\end{frame}
\end{refsection}

\begin{frame}{What did change in the ISO 690 standard}
\framesubtitle{Throughout the editions}
\begin{itemize}
  \item New kinds of information resources
    \begin{itemize}
    \item cartographic materials
    \item audiovisual works
    \item etc.
    \end{itemize}
  \item Explicitly stated, that this International Standard does not
        prescribe a particular style of reference or citation +
        consistency principle etc.
  \item Some terms missing, eg. \emph{chapter}
  \item More detailed description of the principles for creating
        bibliography entries
    \begin{itemize}
      \item changed or added
    \end{itemize}
\end{itemize}
\end{frame}

\begin{frame}{Order of elements}
\framesubtitle{The usual order of elements in a reference}
\begin{description}[name(s) of creator(s)]
\item[name(s) of creator(s)] if available
\item[date] in case of the \emph{author-date} is being applied
\item[title] of the resource
\item[medium designation] if necessary
\item[edition] required if the item is not a first edition
\item[place] the most prominent
\item[publisher] the most prominent
\end{description}
\end{frame}

\begin{frame}{Order of elements}
\framesubtitle{The usual order of elements in a reference}
\begin{description}[additional information]
\item[date] in case of the \emph{author-date} is being applied
            it is not repeated unless a fuller date is necessary
\item[series] title and number
\item[numeration] within the item
\item[standard identifier(s)] ISBN, ISSN, DOI,\dots
\item[availability] access or location information
\item[additional information] pertinent for the item
\end{description}
\end{frame}


\section{The package biblatex-iso690}

\begin{frame}{Already existing solutions}
\framesubtitle{Or what was missing\dots}

\texttt{czechiso.bst}\\
\texttt{*.csl}-based\\
\texttt{OPmac-bib} plainTeX\\
\\
\texttt{BibLaTeX}

\end{frame}

\begin{frame}{Original implementation of \textsf{biblatex-iso690}}
\framesubtitle{By Michal Hoftich}
\begin{itemize}
\item created in 2011
\item by Michal Hoftich
\item build on top of the BibLaTeX package
\item influenced by the Czech national interpretations
\item \url{https://github.com/michal-h21/biblatex-iso690}
\end{itemize}
\end{frame}

\begin{frame}{Original implementation of \textsf{biblatex-iso690}\footnote[frame]{\url{https://github.com/michal-h21/biblatex-iso690}}}
\framesubtitle{By Michal Hoftich}
State of the art
\begin{itemize}
\item adhered to out-of-date editions of the standard % edícia (nepovinné), ' : ' (oddeľovač prvkov, nie interpunkčné znamienko), anon (nemá sa používať termín "Anonym"), miesto vydania + nakladateľ (nepovinné)
\item incorrect order of elements % štandardné identifikátory, číslovanie a stránkovanie
\item (redundant/missing) punctuation % redundantná v prípade chýbajúcich údajov
\item missing some types of resources % patent, thesis
\item missing some required elements % edícia + číslo edície
\item missing creator secondary responsibility % editori, prekladatelia
\item obsolete and deprecated code % nedodržiavanie aktuálnych konvencií biblatexu
\end{itemize}
\end{frame}

\begin{frame}{Reimplementation}
\framesubtitle{Of the package \textsf{biblatex-iso690}}
Macros, commands and definitions
\begin{itemize}
\item printing out the elements in the correct order
\item parsing and formatting of the bibliography fields
\item multilingual references
\item further adjustable
\end{itemize}

From the user perspective
\begin{itemize}
\item more entry types and fields
\item more localisation strings
\item more package options
\end{itemize}
\end{frame}

\begin{frame}{Printing out the elements in the correct order}
\framesubtitle{Original and up-to-date version comparison}
\dots Title~[online]. \textcolor{Red}{Series number.}
      Place: Publisher, 2016 [visited on 2016-11-08], pp. 31--47.
      \textcolor{Red}{Series title.} Url: <path>. ISBN
\[
\Downarrow
\]
\dots Title~[online]. Place: Publisher, 2016, pp. 31--47
      [visited on 2016-11-08]. \textcolor{Green}{Series title,
      series number.} ISBN. Available from~<url>
\end{frame}

\begin{frame}[fragile]{Parsing and formatting of the bibliography fields}
\begin{minted}{latex}
\DeclareFieldFormat{howpublished}{\mkbibbrackets{#1}}
\DeclareFieldFormat[online]{howpublished}{%
  \iffieldundef{howpublished}
    {\mkbibbrackets{online}}
    {\mkbibbrackets{#1}}
}
\end{minted}
\end{frame}

\begin{frame}{Multilingual references}
Main document language
\begin{itemize}
\item language in which the document is written
\item \texttt{babel} package %% \cite babel documentation
  \begin{itemize}
  \item \cmd{\usepackage[dutch,english]{babel}}
  \item \cmd{\usepackage[main=english,dutch]{babel}}
  \end{itemize}
\item \texttt{polyglossia} package
  \begin{itemize}
  \item \cmd{\usepackage{polyglossia}}\\\cmd{\setmainlanguage{english}}
  \end{itemize}
\end{itemize}

Bibliography entry language
\begin{itemize}
\item language of the cited information resource itself
\item \texttt{.bib} file: \cmd{langid = {english}}
\end{itemize}
\end{frame}

% \begin{frame}{Multilingual references}
% \uv{\emph{\dots údaj by mal byť uvedený v~podobe, v~akej sa objavuje v~preferovanom prameni informácií citovaného dokumentu\dots}}
% \newline
% \end{frame}

\begin{frame}{Multilingual references}
Main document language
\begin{itemize}
\item availability and access %% \hfill dostupné z/available from
\item date of citation %% \hfill cit./visited on
\item creator roles %% \hfill ilustroval/ilustrated by
\item conjunctions %% \hfill a/and
\end{itemize}

Bibliography entry language
\begin{itemize}
\item volume %% \hfill ročník/volume
\item edition %% \hfill vyd./ed.
\item numeration %% \hfill č./no.
\end{itemize}
\end{frame}

\begin{frame}{More entry types and fields}
\begin{table}[!htbp]
\begin{tabularx}{\textwidth}{llX}
  %\toprule
  \textbf{Entry field} & \textbf{Entry type} & \textbf{Description}\\
  \midrule
  \texttt{supervisor}     & \texttt{thesis} & advisor of the final qualification work\\
  \texttt{classification} & \texttt{patent} & patent classification\\
  \bottomrule
\end{tabularx}
\caption{Added fields for the bibliography entries}
\label{tab:newfields}
\end{table}
\end{frame}

\begin{frame}{More localisation strings}
\begin{table}[!htbp]
\begin{tabularx}{\textwidth}{lXXX}
  %\toprule
  \textbf{String} & \textbf{English} & \textbf{Czech} & \textbf{Slovak}\\
  \midrule
  \texttt{at}           & {at} & {v} & {v}\\
  \texttt{bysupervisor} & {supervised by} & {vedouc\'{i} pr\'{a}ce} & {ved\'{u}ci pr\'{a}ce}\\
  \texttt{urlalso}      & {available also from} & {dostupn\'{e} tak\'{e} z} & {dostupn\'{e} tie\v{z} z}\\
  \bottomrule
\end{tabularx}
\caption{More localisation strings}
\label{tab:newl10n}
\end{table}
\end{frame}

\begin{frame}{More package options for formatting the output}
\begin{itemize}
  \item \cmd{spacecolon=[true|false]}
    \begin{itemize}
    \item Place : Publisher
    \item Place: Publisher
    \end{itemize}
  \item \cmd{pagetotal=[true|false]}
    \begin{itemize}
    \item Place : Publisher, 2008 [60 p.]
    \item Place : Publisher, 2008
    \end{itemize}
  \item \cmd{shortnumeration=[true|false]}
    \begin{itemize}
    \item \dots\ 2011, \textbf{32}(3), 289--301 [visited on 2016-05-14] \dots
    \item \dots\ 2011, vol. 32, no. 3, pp. 289--301 [visited on 2016-05-14] \dots
    \end{itemize}
  \item \cmd{thesisinfoinnotes=[true|false]}
  \begin{itemize}
    \item \dots\ Available from: <\dots>. BP. MU, FI, Brno. Supervisor Petr SOJKA
    \item \dots\ BP. MU, FI, Brno. Supervisor Petr SOJKA. Available from: <\dots>
    \end{itemize}
\end{itemize}
\end{frame}

\begin{frame}{Redefining in a document preamble}
\begin{itemize}
\item \cmd{\renewcommand\multinamedelim{\addcomma\space}}
  \begin{itemize}
  \item SURNAME, Name\underline{, }SURNAME, Name
  \end{itemize}
\item \cmd{\let\lastnameformat=\textsc}
\item etc.
\end{itemize}

\end{frame}

\section{Examples}

\begin{frame}[fragile]{MWE of biblatex-iso690}
\begin{minted}{latex}
\documentclass{...}
\usepackage[style=iso-authoryear]{biblatex}
\addbibresource{<bibliography-database.bib>}
...
\begin{document}
\parencite{...}
\parencite*{...}
...
\printbibliography
\end{document}
\end{minted}
\end{frame}

\begin{frame}{Sample}
\framesubtitle{Different citation methods and missing fields}
\textbf{author-year: author and editor}\\
\uv{\emph{[\dots] citation of this work~(\textcolor{Dandelion}{Smith}, 2016) [\dots]}}\\
SMITH, John, \textcolor{Green}{2016}. \emph{Masterpiece}. \textcolor{Red}{Ed. DOE, Jane}. Place: Publisher

\rule{\textwidth}{0.2pt}

\textbf{author-year: editor only}\\
\uv{\emph{[\dots] citation of this work~(\textcolor{Dandelion}{Doe}, 2016) [\dots]}}\\
\textcolor{Red}{DOE, Jane (ed.)}, \textcolor{Green}{2016}. \emph{Masterpiece}. Place: Publisher

\rule{\textwidth}{0.2pt}

\textbf{numeric style}\\
\uv{\emph{[\dots] citation of this work~[1] [\dots]}}\\
SMITH, John. \emph{Masterpiece}. Ed. DOE, Jane. Place: Publisher, \textcolor{Green}{2016}
\end{frame}

\begin{frame}[fragile]{More samples}
\fullcite{Ryan1998}
\newline
\begin{minted}[fontsize=\scriptsize]{latex}
@misc{Ryan1998,
  title = {Saving Private Ryan},
  howpublished = {DVD},
  editora = {Steven Spielberg},
  editoratype = {Directed by},
  publisher = {Paramount},
  date = {1998}
}
\end{minted}
\end{frame}

\begin{frame}[fragile]
\fullcite{LibreOfficeLatest}
\newline
\begin{minted}[fontsize=\scriptsize]{latex}
@misc{LibreOfficeLatest,
  author = {Ruediger Timm and others},
  title = {Libre Office},
  howpublished = {software},
  publisher = {The Document Foundation},
  date = {2000/2016},
  version = {5.2.3},
  url = {https://www.libreoffice.org/download/libreoffice-fresh},
  note = {System requirements: Linux kernel version 2.6.18 or higher...}
}
\end{minted}
\end{frame}

\begin{frame}[fragile]
\fullcite{Groll2008}
\newline
\begin{minted}[fontsize=\scriptsize]{latex}
@patent{Groll2008,
  author = {Clad Metals LLC Canonsburg, PA 15317 (US)},
  title = {Method of making a copper core five-ply composite and...},
  editora = {Groll, W.A.},
  editoratype = {Inventor:},
  publisher = {Google Patents},
  classification = {EP 1 094 937 B1},
  type = {patenteu},
  url = {https://encrypted.google.com/patents/EP1094937B1}
}
\end{minted}
\end{frame}

% \begin{frame}[fragile]
% \fullcite{Luptak2016}
% \newline
% \begin{minted}[fontsize=\scriptsize]{tex}
% @thesis{Luptak2016,
% AUTHOR = "Lupták, Dávid",
% TITLE = "Sazba bibliografie dle normy ISO 690",
% HOWPUBLISHED = "online",
% URLDATE = "2016-06-14",
% TYPE = "Bakalářská práce",
% SCHOOL = "Masarykova univerzita, Fakulta informatiky, Brno",
% SUPERVISOR = "Petr Sojka",
% URL = "http://is.muni.cz/th/422640/fi_b/",
% }
% \end{minted}
% \end{frame}

\section{Summary}

\begin{frame}{Summary}
\begin{itemize}
\item[\checkmark] full reimplementation of the \textsf{biblatex-iso690} package
\item[\checkmark] compliant with the latest edition of the standard ISO 690
\item[\checkmark] Czech and English documentation
  \begin{itemize}
    \item use case examples
    \item hints and caveats
  \end{itemize}
\item[\checkmark] first public release
  \begin{itemize}
  \item integrated into the theses document class
        \textsf{fithesis3}\footnote[frame]{\url{https://www.ctan.org/pkg/fithesis}}
  \item published on CTAN\footnote[frame]{\url{https://www.ctan.org/pkg/biblatex-iso690}}
    \begin{itemize}
    \item part of \TeX\ Live 2016
    \end{itemize}
  \end{itemize}
\end{itemize}
\end{frame}

\begin{frame}{Summary}
\begin{itemize}
  \item[\checkmark] Slovak localisation for the \textsf{biblatex}\footnote[frame]{\url{https://www.ctan.org/pkg/biblatex}} package
  \item[\checkmark] pull requests by the community (e.g. translations)
  \item[\checkmark] 51 stars of the project in the GitHub repository
  \item[\checkmark] 17 forks of the project in the GitHub repository
\end{itemize}
\end{frame}

% \begin{frame}{Čo ďalej}
% \begin{itemize}
%   \item diskusia o~rozšírení balíka o~metódy citovania mimo ISO 690
%   \item reakcia na dynamicky sa vyvýjajúci balík BibLaTeX
%   \item rozšírenie dokumentácie (príp. wiki) o~\textit{style guide}
%   \item \dots
% \end{itemize}
% \end{frame}


\begin{darkframes}
\begin{frame}[plain]{%
  \mbox{}
  \newcounter{thanksFrameNumber}
  \setcounter{thanksFrameNumber}{\value{framenumber}}
  \title{Thanks for your attention!}
  \subtitle{Q\&A}
  \titlepage
  \setcounter{framenumber}{\value{thanksFrameNumber}}
}
\end{frame}
\end{darkframes}

\end{document}

% simple _template_ for q&a
\begin{frame}{}
\emph{}
\medskip\par

\end{frame}
