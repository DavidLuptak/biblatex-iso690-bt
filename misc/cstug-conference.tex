\PassOptionsToPackage{dvipsnames}{xcolor} % have more already defined colors
\documentclass{beamer}
\usetheme[
  university=mu,
  faculty=fi,
  logoLocale=czech,
]{fibeamer}

%\usecolortheme{crane}
\setbeamercovered{transparent} %% Watermarked items..

\AtBeginSection[] % Do nothing for \section*
{
\begin{frame}<beamer>
%\frametitle{Outline}
\tableofcontents[currentsection]
\end{frame}
}

\usepackage[utf8]{inputenc}
\usepackage[slovak]{babel}
\usepackage{amssymb}  %% Typesetting \checkmark
\usepackage{minted}  %% Highlighted source code for LaTeX
\usepackage{tabularx}  %% Tabulars with adjustable-width columns
\usepackage{booktabs}  %% Publication quality tables in LaTeX -- \toprule, etc.

\newcommand{\cmd}[1]{\mintinline{latex}{#1}}

\usepackage{csquotes}  %% Context sensitive quotation facilities
\usepackage[style=iso-numeric,shortnumeration,pagetotal,thesisinfoinnotes=false]{biblatex}

\addbibresource{reference_resource.bib}

% Demo showing how the iso style could be easily switched
% \url{https://www.overleaf.com/6955878njcdygbdnggw}

\author{Dávid Lupták}
\title{Sadzba bibliografie podľa normy ISO 690 v~systéme LaTeX}
\subtitle{Valná hromada CSTUG 2016}
% \date{2016-12-17}
% \institute[FI MU]{Fakulta informatiky\\Masarykova Univerzita, Brno}

%%%%%%%%%%%%%%%%%%%%%%%%%%%%%%%%%%
% biblatex environment adjustment
\DeclareFieldFormat{labelnumberwidth}{\mkbibbrackets{#1}}

\renewcommand\MethodFormat{%
  \printtext[labelnumberwidth]{%
    \printfield{labelprefix}%
    \printfield{labelnumber}}}%

\defbibenvironment{bibliography}
  {\list%
     {\MethodFormat}%
     {\setlength{\leftmargin}{7.5mm}%
      \setlength{\itemindent}{0mm}%
      \setlength{\itemsep}{\bibitemsep}%
      \setlength{\parsep}{\bibparsep}}%
      \renewcommand*{\makelabel}[1]{\hss##1}
      %\raggedright}
      }%
  {\endlist}%
  {\item}%
%%%%%%%%%%%%%%%%%%%%%%%%%%%%%%%%%%

% reset counter for every section
\makeatletter
\@addtoreset{footnote}{section}
\makeatother

\begin{document}

\frame{\titlepage}
\frame{\tableofcontents}

\section{Norma ISO 690}

\begin{frame}{Norma ISO 690}
Informace a dokumentace -- Pravidla pro bibliografické odkazy a citace informačních zdrojů
\\
\\
Vydaná Medzinárodnou organizáciou pre normalizáciu (ISO)
\\
\\
Jedna z~najčastejších pravidiel/metodických pokynov pre citovanie.
\end{frame}

\begin{frame}{Norma ISO 690}
\framesubtitle{Terminológia}
\begin{description}[citácia]
\item[odkaz] údaj v~texte alebo iný druh obsahu dokumentu na príslušnú bibliografickú citáciu
\item[citácia] dáta popisujúce informačný zdroj alebo jeho časť, dostatočne presne a podrobne na to, aby mohol byť tento zdroj identifikovaný a bolo možné ho vyhľadať
\end{description}
\end{frame}

\begin{refsection} % biblatex feature to separate bibliographies
\begin{frame}{Terminológia na príklade}
\framesubtitle{Podľa najnovšej verzie normy ČSN ISO 690:2011}
\uv{\emph{[\dots] aj netriviálne deduktívne hry môžu byť vyriešené softvérovým nástrojom~\cite{Klimos2015minimal}, čo vedie k~záveru [\dots]}}
\rule{\textwidth}{0.2pt}
\printbibliography
\end{frame}
\end{refsection}

\begin{frame}{Čo sa zmenilo v~norme ISO 690}
\framesubtitle{Oproti predchádzajúcim verziám}
\[
\underbrace{\phantom{\text{-2}}\strut\text{\Large{ISO 690:1987 + ISO 690-2:1997}}}_\text{}
\]
\begin{center}
\LARGE{ISO 690:2010}
\end{center}
\begin{center}
\textcolor{Gray}{ČSN ISO 690:2011, STN ISO 690:2012, DIN ISO 690:2013, etc.}
\end{center}
\end{frame}

\begin{frame}{Čo sa zmenilo v~norme ISO 690}
\framesubtitle{Oproti predchádzajúcim verziám}
\begin{itemize}
  \item Nové druhy informačných zdrojov
    \begin{itemize}
    \item kartografické materiály
    \item audiovizuálne diela
    \item etc.
    \end{itemize}
  \item Explicitné zdelenie, že norma nepredpisuje konkrétny štýl odkazu alebo citácie
  \item Chýbajúce termíny, napr. \emph{kapitola}
  \item Podrobnejší výklad pravidiel tvorby jednotlivých bibl. údajov
    \begin{itemize}
      \item zmenené alebo doplnené
    \end{itemize}
\end{itemize}
\end{frame}

\begin{frame}{Poradie prvkov citácie}
\framesubtitle{Doporučené všeobecné poradie prvkov citácií}
\begin{description}[miesto vydania]
\item[mená tvorcov] ak sú k~dispozícii
\item[rok] v~prípade citovania metódou \emph{autor-dátum}
\item[názov] citovanej jednotky
\item[typ nosiča] v~prípade potreby
\item[vydanie] okrem prvého
\item[miesto vydania] v~prípade viacerých rovnako významných miest sa z~nich uvádza len prvé
\item[vydavateľ] v~prípade viacerých rovnako významných vydavateľov sa z~nich uvádza len prvý
\end{description}
\end{frame}

\begin{frame}{Poradie prvkov citácie}
\framesubtitle{Doporučené všeobecné poradie prvkov citácií}
\begin{description}[štandardné identifikátory]
\item[rok] v~prípade citovania metódou \emph{autor-dátum} sa už na tomto mieste neopakuje, s~výnimkou uvedenia presnejšieho dátumu
\item[edícia] názov a číslo
\item[číslovanie] v~rámci popisovanej jednotky
\item[štandardné identifikátory] ISBN, ISSN, DOI,\dots
\item[dostupnosť] prístup alebo umiestnenie
\item[dodatočné informácie] relevantné pre účely špecifickej citácie
\end{description}
\end{frame}

% \begin{frame}{Nejasné miesta a nepresnosti}

% \end{frame}

% \begin{frame}{Problémy}

% \end{frame}

\section{Balík biblatex-iso690}

\begin{frame}{Existujúce riešenia}
\framesubtitle{Alebo čo chýbalo\dots}

\texttt{czechiso.bst}\\
\texttt{*.csl}-based\\
\texttt{OPmac-bib} plainTeX\\
\\
\texttt{biblatex-iso690}

\end{frame}

\begin{frame}{Referenčná implementácia \textsf{biblatex-iso690}}
\framesubtitle{Autorom Michal Hoftich}
\begin{itemize}
\item vznik v~roku 2011
\item autorom Michal Hoftich
\item využitie balíka BibLaTeX
\item pridržiavanie sa českých národných interpretácií
\item \url{https://github.com/michal-h21/biblatex-iso690}
\end{itemize}
\end{frame}

\begin{frame}{Referenčná implementácia \textsf{biblatex-iso690}\footnote[frame]{\url{https://github.com/michal-h21/biblatex-iso690}}}
\framesubtitle{Autorom Michal Hoftich}
Pôvodný stav
\begin{itemize}
\item pridržiavanie sa predošlých verzií noriem % edícia (nepovinné), ' : ' (oddeľovač prvkov, nie interpunkčné znamienko), anon (nemá sa používať termín "Anonym"), miesto vydania + nakladateľ (nepovinné)
\item chybné poradie prvkov citácie % štandardné identifikátory, číslovanie a stránkovanie
\item (nadbytočná/chýbajúca) interpunkcia % redundantná v prípade chýbajúcich údajov
\item chýbajúca podpora niektorých typov dokumentov % patent, thesis
\item chýbajúca podpora niektorých vyžadovaných prvkov citácie % edícia + číslo edície
\item chýbajúca sekundárna zodpovednosť % editori, prekladatelia
\item zastaraný kód % nedodržiavanie aktuálnych konvencií biblatexu
\end{itemize}
\end{frame}

\begin{frame}{Kompletná reimplementácia}
\framesubtitle{Balíka \textsf{biblatex-iso690}}
Makrá, príkazy a definície
\begin{itemize}
\item výpis jednotlivých elementov v~správnom poradí
\item spracovávanie a formátovanie bibliografických údajov
\item viacjazyčnosť citácie
\item ďalej prispôsobiteľné
\end{itemize}

Z~užívateľského pohľadu
\begin{itemize}
\item nové typy a polia záznamov
\item nové lokalizačné reťazce
\item nové voľby balíka na zmenu výstupného formátu
\end{itemize}
\end{frame}

\begin{frame}{Výpis jednotlivých elementov v~správnom poradí}
\framesubtitle{Citácia vygenerovaná pôvodnou a aktuálnou verziou}
\dots Názov~[online]. \textcolor{Red}{Č. edície.} Miesto: Nakladateľstvo, 2016 [cit. 2016-11-08], s. 31--47. \textcolor{Red}{Edícia.} Url: <cesta>. ISBN
\[
\Downarrow
\]
\dots Názov~[online]. Miesto: Nakladateľstvo, 2016, s. 31--47 [cit. 2016-11-08]. \textcolor{Green}{Edícia, č. edície.} ISBN. Dostupné z~<url>
\end{frame}

\begin{frame}[fragile]{Spracovávanie a formátovanie bibliografických údajov}
\begin{minted}{latex}
\DeclareFieldFormat{howpublished}{\mkbibbrackets{#1}}
\DeclareFieldFormat[online]{howpublished}{%
  \iffieldundef{howpublished}
    {\mkbibbrackets{online}}
    {\mkbibbrackets{#1}}
}
\end{minted}
\end{frame}

\begin{frame}{Viacjazyčnosť citácie}
Hlavný jazyk dokumentu
\begin{itemize}
\item jazyk, v~ktorom je dokument písaný
\item balík \texttt{babel} %% \cite babel documentation
  \begin{itemize}
  \item \cmd{\usepackage[dutch,english]{babel}}
  \item \cmd{\usepackage[main=english,dutch]{babel}}
  \end{itemize}
\item balík \texttt{polyglossia}
  \begin{itemize}
  \item \cmd{\usepackage{polyglossia}}\\\cmd{\setmainlanguage{english}}
  \end{itemize}
\end{itemize}

Jazyk bibliografického záznamu
\begin{itemize}
\item jazyk prameňa informácií citovaného dokumentu
\item \texttt{.bib} súbor: \cmd{langid = {english}}
\end{itemize}
\end{frame}

\begin{frame}{Viacjazyčnosť citácie}
Hlavný jazyk dokumentu
\begin{itemize}
\item informácie o~dostupnosti %% \hfill dostupné z/available from
\item dátum citovania %% \hfill cit./visited on
\item role tvorcov %% \hfill ilustroval/ilustrated by
\item spojky %% \hfill a/and
\end{itemize}

Jazyk bibliografického záznamu
\begin{itemize}
\item ročník %% \hfill ročník/volume
\item vydanie %% \hfill vyd./ed.
\item číslovanie %% \hfill č./no.
\end{itemize}
\end{frame}

\begin{frame}{Nové typy a polia záznamov}
\begin{table}[!htbp]
\begin{tabularx}{\textwidth}{llX}
  %\toprule
  \textbf{Názov poľa} & \textbf{Typ záznamu} & \textbf{Popis}\\
  \midrule
  \texttt{supervisor}     & \texttt{thesis} & vedúci záverečnej kvalifikačnej práce\\
  \texttt{classification} & \texttt{patent} & klasifikácia podľa patentového triedenia\\
  \bottomrule
\end{tabularx}
\caption{Nové polia pre bibliografické údaje}
\label{tab:newfields}
\end{table}
\end{frame}

\begin{frame}{Nové lokalizačné reťazce}
\begin{table}[!htbp]
\begin{tabularx}{\textwidth}{lXXX}
  %\toprule
  \textbf{Reťazec} & \textbf{Anglicky} & \textbf{Česky} & \textbf{Slovensky}\\
  \midrule
  \texttt{at}           & {at} & {v} & {v}\\
  \texttt{bysupervisor} & {supervised by} & {vedouc\'{i} pr\'{a}ce} & {ved\'{u}ci pr\'{a}ce}\\
  \texttt{urlalso}      & {available also from} & {dostupn\'{e} tak\'{e} z} & {dostupn\'{e} tie\v{z} z}\\
  \bottomrule
\end{tabularx}
\caption{Nové lokalizačné reťazce}
\label{tab:newl10n}
\end{table}
\end{frame}

\begin{frame}{Nové voľby balíka na zmenu výstupného formátu}
\begin{itemize}
  \item \cmd{spacecolon=[true|false]}
    \begin{itemize}
    \item Miesto : Nakladateľstvo
    \item Miesto: Nakladateľstvo
    \end{itemize}
  \item \cmd{pagetotal=[true|false]}
    \begin{itemize}
    \item Miesto: Nakladateľstvo, 2008 [60 s.]
    \item Miesto: Nakladateľstvo, 2008
    \end{itemize}
  \item \cmd{shortnumeration=[true|false]}
    \begin{itemize}
    \item \dots\ 2011, \textbf{32}(3), 289--301 [cit. 2016-05-14] \dots
    \item \dots\ 2011, roč. 32, č.3, s. 289--301 [cit. 2016-05-14] \dots
    \end{itemize}
  \item \cmd{thesisinfoinnotes=[true|false]}
  \begin{itemize}
    \item \dots\ Dostupné z: <\dots>. BP. MU, FI, Brno. Vedúci práce Petr SOJKA
    \item \dots\ BP. MU, FI, Brno. Vedúci práce Petr SOJKA. Dostupné z: <\dots>
    \end{itemize}
\end{itemize}
\end{frame}

\section{Príklady}

\begin{frame}[fragile]{MWE pre biblatex-iso690 obecne}
\begin{minted}{latex}
\documentclass{...}
\usepackage[style=iso-authoryear]{biblatex}
\addbibresource{<bibliografická-databáza.bib>}
...
\begin{document}
\parencite{...}
\parencite*{...}
...
\printbibliography
\end{document}
\end{minted}
\end{frame}

\begin{frame}{Ukážka}
\framesubtitle{Rôzne metódy citovania a absencia údajov}
\textbf{autor-dátum: autor aj editor}\\
\uv{\emph{[\dots] príslušný odkaz v~texte na toto dielo~(\textcolor{Dandelion}{Kováč}, 2016) [\dots]}}\\
KOVÁČ, Peter, \textcolor{Green}{2016}. \emph{Veľdielo}. \textcolor{Red}{Ed. NOVÁK, Ján}. Miesto: Nakladateľstvo

\rule{\textwidth}{0.2pt}

\textbf{autor-dátum: iba editor}\\
\uv{\emph{[\dots] príslušný odkaz v~texte na toto dielo~(\textcolor{Dandelion}{Novák}, 2016) [\dots]}}\\
\textcolor{Red}{NOVÁK, Ján (ed.)}, \textcolor{Green}{2016}. \emph{Veľdielo}. Miesto: Nakladateľstvo

\rule{\textwidth}{0.2pt}

\textbf{číslované odkazy}\\
\uv{\emph{[\dots] príslušný odkaz v~texte na toto dielo~[1] [\dots]}}\\
KOVÁČ, Peter. \emph{Veľdielo}. Ed. NOVÁK, Ján. Miesto: Nakladateľstvo, \textcolor{Green}{2016}
\end{frame}

\begin{frame}[fragile]{Ukážky vybraných typov záznamov}
\fullcite{Ryan1998}
\newline
\begin{minted}[fontsize=\scriptsize]{latex}
@misc{Ryan1998,
  title = {Saving Private Ryan},
  howpublished = {DVD},
  editora = {Steven Spielberg},
  editoratype = {Réžia},
  publisher = {Paramount},
  date = {1998}
}
\end{minted}
\end{frame}

\begin{frame}[fragile]
\fullcite{LibreOfficeLatest}
\newline
\begin{minted}[fontsize=\scriptsize]{latex}
@misc{LibreOfficeLatest,
  author = {Ruediger Timm and others},
  title = {Libre Office},
  howpublished = {softvér},
  publisher = {The Document Foundation},
  date = {2000/2016},
  version = {5.2.3},
  url = {https://www.libreoffice.org/download/libreoffice-fresh},
  note = {Požiadavky na systém: Linux kernel verzie 2.6.18 alebo vyššej...}
}
\end{minted}
\end{frame}

\begin{frame}[fragile]
\fullcite{Groll2008}
\newline
\begin{minted}[fontsize=\scriptsize]{latex}
@patent{Groll2008,
  author = {Clad Metals LLC Canonsburg, PA 15317 (US)},
  title = {Method of making a copper core five-ply composite and...},
  editora = {Groll, W.A.},
  editoratype = {Vynálezca:},
  publisher = {Google Patents},
  classification = {EP 1 094 937 B1},
  type = {patenteu},
  url = {https://encrypted.google.com/patents/EP1094937B1}
}
\end{minted}
\end{frame}

\begin{frame}[fragile]
\fullcite{Luptak2016}
\newline
\begin{minted}[fontsize=\scriptsize]{tex}
@thesis{Luptak2016,
AUTHOR = "Lupták, Dávid",
TITLE = "Sazba bibliografie dle normy ISO 690",
HOWPUBLISHED = "online",
URLDATE = "2016-06-14",
TYPE = "Bakalářská práce",
SCHOOL = "Masarykova univerzita, Fakulta informatiky, Brno",
SUPERVISOR = "Petr Sojka",
URL = "http://is.muni.cz/th/422640/fi_b/",
}
\end{minted}
\end{frame}

\section{Zhrnutie}

\begin{frame}{Zhrnutie}
\begin{itemize}
\item[\checkmark] kompletná reimplementácia referenčnej implementácie balíka \textsf{biblatex-iso690}
\item[\checkmark] spĺňanie pravidiel normy ISO 690
\item[\checkmark] dokumentácia v~českom i anglickom jazyku
  \begin{itemize}
    \item ukážky použitia
    \item rady a varovania
  \end{itemize}
\item[\checkmark] oficiálne začlenenie a vydanie štýlu
  \begin{itemize}
  \item do šablóny na sadzbu záverečných kvalifikačných prác \textsf{fithesis3}\footnote[frame]{\url{https://www.ctan.org/pkg/fithesis}}
  \item na medzinárodný archív CTAN\footnote[frame]{\url{https://www.ctan.org/pkg/biblatex-iso690}}
    \begin{itemize}
    \item súčasťou \TeX\ Live 2016
    \end{itemize}
  \end{itemize}
\end{itemize}
\end{frame}

\begin{frame}{Zhrnutie}
\begin{itemize}
  \item[\checkmark] rozšírenie balíka \textsf{biblatex}\footnote[frame]{\url{https://www.ctan.org/pkg/biblatex}} o~slovenskú lokalizáciu
  \item[\checkmark] pribúdajúce pull requesty od komunity (napr. jazykové lokalizácie)
  \item[\checkmark] 47 označení \uv{obľúbené} projektu na GitHub repozitári
  \item[\checkmark] 17 forkov projektu na GitHub repozitári
\end{itemize}
\end{frame}

\begin{frame}{A čo ďalej}
\begin{itemize}
  \item diskusia o~rozšírení balíka o~metódy mimo ISO 690
  \item reakcia na dynamicky sa vyvýjajúci balík BibLaTeX
  \item rozšírenie dokumentácie (príp. wiki) o~\textit{style guide}
  \item \dots
\end{itemize}
\end{frame}


\begin{darkframes}
\begin{frame}[plain]{%
  \mbox{}
  \newcounter{thanksFrameNumber}
  \setcounter{thanksFrameNumber}{\value{framenumber}}
  \title{Ďakujem za pozornosť!}
  \subtitle{Q\&A}
  \titlepage
  \setcounter{framenumber}{\value{thanksFrameNumber}}
}
\end{frame}
\end{darkframes}

\end{document}

% simple _template_ for q&a
\begin{frame}{}
\emph{}
\medskip\par

\end{frame}
