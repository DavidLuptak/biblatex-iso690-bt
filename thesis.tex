\documentclass{fithesis3}
\thesissetup{
	faculty=fi
	,title=Sadzba bibliografie podľa normy ISO 690
	,author=Dávid Lupták
	,advisor={Doc.\ RNDr.\ Petr Sojka,\ Ph.D.}
	,keywords={ISO-690:2011, Bib\LaTeX}}
\usepackage[slovak]{babel}
\begin{document}
\chapter{Úvod}

\chapter{Implementácie normy ISO 690}
Dodržiavanie normy ISO 690 patrí medzi najčastejšie doporučenia pri tvorbe citácií a bibliografických záznamov v rámci kvalifikačných prác v akademickom prostredí. Preto vzniklo už niekoľko implementácií pre rôzne programy a nástroje, ktoré využitie tejto normy podporujú. V tejto kapitole autor popíše existujúce programy a nástroje implementujúce normu ISO 690. Hoci nepôjde o implementácie v rámci balíka Bib\LaTeX, aj napriek tomu nám tento popis poskytne lepší náhľad do problematiky týkajúcej sa normy ISO 690.

	\section{Citace.com}
	Portál citace.com je online služba zameraná na automatické generovanie a správu citácií. Bola vytvorená Martinom Krčálom v roku ??. V základnej verzii služba vychádza z normy ČSN ISO 690 z roku 2011. Norma je implementovaná pomocou programovacieho jazyka Citation Style Language (CSL).
	
\end{document}
