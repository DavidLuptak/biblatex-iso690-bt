\documentclass{fithesis3}
\thesissetup{
	faculty=fi
	,title=Sadzba bibliografie podľa normy ISO 690
	,author=Dávid Lupták
	,advisor={Doc.\ RNDr.\ Petr Sojka,\ Ph.D.}
	,keywords={ISO 690:2011, Bib\LaTeX}}
\usepackage[slovak]{babel}
\begin{document}
\chapter{Úvod}

\chapter{Implementácie normy ISO 690}
Dodržiavanie normy ISO 690 patrí medzi najčastejšie doporučenia pri tvorbe citácií a bibliografických záznamov v rámci kvalifikačných prác v akademickom prostredí. Preto vzniklo už niekoľko implementácií pre rôzne programy a nástroje, ktoré využitie tejto normy podporujú. V tejto kapitole autor popíše existujúce programy a nástroje implementujúce normu ISO 690. Hoci nepôjde o implementácie v rámci balíka Bib\LaTeX, aj napriek tomu nám tento popis poskytne lepší náhľad do problematiky týkajúcej sa normy ISO 690.

	\section{Citace.com}
	Portál citace.com patrí medzi najpoužívanejšie generátory citácií v Českej republike. [cite -- naucte sa citovat] Ide o voľne dostupnú online službu generujúcu citácie podľa normy ČSN ISO 690. V rámci veľkého množstva online generátorov citácií predčí ostatné práve svojou jednoduchosťou používania, čomu zodpovedá aj motto portálu "...citovat je snadné".

	Samotný projekt citace.com vznikol v roku 2003, pôvodne ako študentský projekt pod názvom Bibliografické citácie. V súčasnosti je však skôr známy už len ako citace.com. Autorom je Martin Krčál, ktorého bezpochyby možno označiť za duchovného otca projektu. Vedúcim projektu je dodnes a jeho aktuálnou pracovnou náplňou je metodické vedenie a riadenie projektu. Veľmi rýchlo po vzniku projektu sa služba citace.com dostala do povedomia študentov na Masarykovej univerzite, ale aj na iných vysokých školách v Českej republike a na Slovensku. Postupne si v univerzitnom prostredí získala relatívne vysoké renomé nielen medzi študentami, ale dokonca aj vyučujúcimi. Dnes patrí medzi najpoužívanejšie generátory citácií v Českej republike.

	Spočiatku mal projekt za cieľ poskytnúť užívateľom jednoduchý nástroj na generovanie citácií. Neskôr sa však z jednoduchého generátora citácií stal citačný softvér. Projekt citace.com totiž okrem bezplatnej verzie generátoru citácií poskytuje aj komerčný produkt s názvom Citace PRO. Ten je určený a vyvíjaný najmä pre inštitúcie, nakoľko ide o komplexný citačný manažér. Okrem základnej funkcionality generovania citácií poskytuje širšiu paletu citačných štýlov, úložisko pre dokumenty, možnosť spolupráce pri tvorbe citácií v rámci danej inštitúcie i mimo nej, podporu inštitucionálnej autentizácie a ďalšie funkcie.

	Služba od svojho vzniku prešla niekoľkými zmenami, čo je v rámci životného cyklu softvéru prirodzená vlastnosť. Musela totiž reagovať nielen na technologický pokrok, ale rovnako aj na potreby užívateľov alebo aktualizáciu samotnej normy ISO 690. Spomedzi množstva funkcií aktuálnej verzie systému si určite zaslúži vyzdvihnúť dohľadávanie citácií, ktoré výrazne zjednodušuje tvorbu citácií. Na vygenerovanie požadovanej citácie užívateľovi stačí zadať len názov diela alebo niektorý z jednoznačných identifikátorov, tj. ISBN alebo DOI. Z knihovných katalógov sa záznam dohľadá a automaticky sa vyplnia príslušné polia vo formulári, ktoré si potom užívateľ môže dodatočne upraviť.

	Výrazným a hlavným podnetom k jednej z najrozsiahlejších zmien projektu bola už spomínaná aktualizácia normy ISO 690. Dôsledkom tejto aktualizácie bolo aj vytvorenie interpretácie normy poprednými odborníkmi na problematiku citácií v Českej republike. Tento dokument slúžil ako podklad pre implementáciu normy na portáli citace.com, zároveň však nadobudol veľkú váhu aj v rámci komunity užívateľov. V základnej verzii teraz služba vychádza práve z tejto normy, tj. normy ČSN ISO 690 z roku 2011.

	V bezplatnej online verzii služby citace.com je jediným citačným štýlom práve citovanie podľa normy ISO 690. V rámci spoplatnenej varianty Citace PRO je však k dispozícii množstvo ďalších štýlov, a to v počte viac ako 8000. Citačné štýly sú implementované pomocou programovacieho jazyka Citation Style Language (CSL). Existuje databáza existujúcich voľne dostupných CSL citačných štýlov, ktorú využíva aj portál citace.com.

	\section{Zotero}
	
\chapter{Balík Bib\LaTeX}
Balík Bib\LaTeX\, aktuálne podporuje niekoľko citačných štýlov. Oficiálnu podporu pre normu ISO 690 však zatiaľ neposkytuje. V tejto kapitole sa preto autor zameria na existujúce riešenia normy ISO 690 pre balík Bib\LaTeX.

	\section{Porovnanie s jazykom CSL}
	\section{Neoficiálny štýl normy ISO 690}

\chapter{Implementácia}
	\section{Odchýlky od normy}

\chapter{Slovenská lokalizácia Bib\LaTeX u}

\chapter{Záver}
\end{document}
